%%%%%%%%%%%%%%%%%%%%%%%%%%%%%%%%%%%%%%%%%%%%%%%%%%%%%%%%%%%%%%%%%%%%%%%%%%%%%%%%%%%%%%%%%%%%%%%%%%%%%%%%%%%%%%%%%%%%%%%%%%%%%%%%%%%%%%%%%%%%%%%%%%%%%%%%%%%%%%%%%%%%%%%%%%%%%%%%%%%%%%%%%%%%
% Written By Michael Brodskiy
% Class: Principles of Macroeconomics
% Professor: H. Yoon
%%%%%%%%%%%%%%%%%%%%%%%%%%%%%%%%%%%%%%%%%%%%%%%%%%%%%%%%%%%%%%%%%%%%%%%%%%%%%%%%%%%%%%%%%%%%%%%%%%%%%%%%%%%%%%%%%%%%%%%%%%%%%%%%%%%%%%%%%%%%%%%%%%%%%%%%%%%%%%%%%%%%%%%%%%%%%%%%%%%%%%%%%%%%

\documentclass[12pt]{article} 
\usepackage{alphalph}
\usepackage[utf8]{inputenc}
\usepackage[russian,english]{babel}
\usepackage{titling}
\usepackage{amsmath}
\usepackage{graphicx}
\usepackage{enumitem}
\usepackage{amssymb}
\usepackage[super]{nth}
\usepackage{everysel}
\usepackage{ragged2e}
\usepackage{geometry}
\usepackage{fancyhdr}
\usepackage{cancel}
\usepackage{siunitx}
\usepackage{xcolor}
\usepackage{physics}
\usepackage{tikz}
\usepackage{mathdots}
\usepackage{yhmath}
\usepackage{color}
\usepackage{array}
\usepackage{multirow}
\usepackage{gensymb}
\usepackage{tabularx}
\usepackage{extarrows}
\usepackage{booktabs}
\usetikzlibrary{fadings}
\usetikzlibrary{patterns}
\usetikzlibrary{shadows.blur}
\usetikzlibrary{shapes}


\geometry{top=1.0in,bottom=1.0in,left=1.0in,right=1.0in}
\newcommand{\subtitle}[1]{%
  \posttitle{%
    \par\end{center}
    \begin{center}\large#1\end{center}
    \vskip0.5em}%

}
\usepackage{hyperref}
\hypersetup{
colorlinks=true,
linkcolor=blue,
filecolor=magenta,      
urlcolor=blue,
citecolor=blue,
}

\urlstyle{same}


\title{Lecture 1 Notes}
\date{June 15, 2021}
\author{Michael Brodskiy\\ \small Instructor: Prof. Yoon}

% Mathematical Operations:

% Sum: $$\sum_{n=a}^{b} f(x) $$
% Integral: $$\int_{lower}^{upper} f(x) dx$$
% Limit: $$\lim_{x\to\infty} f(x)$$

\begin{document}

    \maketitle

    \begin{enumerate}

      \item Economics — The study of how people manage scarce resources to achieve their goals

      \item Economic Groups:

        \begin{enumerate}

          \item Individuals, Households, and Consumers — Consume goods or services and provide factors of production

            \begin{enumerate}

              \item Goals: Maximize utility from consuming goods or services and providing factors of production

              \item Resources: Income to purchase goods or services, time

            \end{enumerate}

          \item Firms, Entrepreneurs, Businesses, Companies, and Producers — Produce goods or services and use factors of production
            
            \begin{enumerate}

              \item Goals: Maximize profit from producing and selling goods or services and using factors of production

              \item Resources: Labor, Natural Resources, Capital, and Entrepreneurship

            \end{enumerate}

          \item Governments — Spend their budget to implement their policies or social welfare programs

            \begin{enumerate}

              \item Goals: Maximize the well-being of consumers and producers

              \item Resources: Budget

            \end{enumerate}

        \end{enumerate}

      \item Three Economic Assumptions:

        \begin{enumerate}

          \item People are rational

            \begin{enumerate}

              \item People make decisions to achieve their highest goals given scarce resources and available information

              \item Economists assume that people are rational on average even though not everyone behaves rationally

              \item Ex. Individuals consume goods or services to maximize their utility given income and prices of goods or services

            \end{enumerate}

          \item People respond to incentives

            \begin{enumerate}

              \item An incentive is something that causes a change in the trade-offs that people face

              \item Incentive involves total benefit and total cost

                \begin{enumerate}

                  \item Total Benefit — Total amount of gain by doing the activity

                  \item Total Cost — Total amount paid or forfeited to do the activity

                \end{enumerate}

              \item Positive incentives makes people do more of an activity, while negative incentives (disincentives) decrease likelihood of performing an activity

            \end{enumerate}

          \item People make marginal decisions

            \begin{enumerate}

              \item Rational people make decisions by comparing marginal benefit to marginal cost

                \begin{enumerate}

                  \item Marginal Benefit is the benefit of performing an additional unit of an activity: $B_M=\frac{TB_1-TB_0}{Q_1-Q_0}$, where $TB_1$ is the new total benefit, $TB_0$ is the old total benefit, $Q_1$ is the new amount of activity, and $Q_0$ is the old amount of activity

                  \item Marginal Cost is the cost of performing an additional unit of an activity: $C_M=\frac{TC_1-TC_0}{Q_1-Q_0}$, where $TC_1$ is the new total cost, $TC_0$ is the old total cost, $Q_1$ is the new amount of activity, and $Q_0$ is the old amount of activity

                \end{enumerate}

              \item Decision criteria depends on how much of an activity people take to maximize net benefit ($TB-TC$)

                \begin{enumerate}

                  \item If $B_M>C_M$, more of an activity should be done to increase net benefit

                  \item If $B_M<C_M$, less of an activity should be done to increase net benefit

                  \item If $B_M=C_M$, the level of activity should be maintained to keep benefit in equilibrium

                \end{enumerate}

              \item This process is known as the \textbf{marginal decision-making process}

            \end{enumerate}

        \end{enumerate}

      \item \textbf{Scarcity}:

        \begin{enumerate}

          \item Peoples' wants are unlimited or infinite, but resources are limited, scarce, or finite

          \item Peoples' wants are constrained by scarce resources

          \item People need to make a choice among alternatives

          \item Ex. Individuals need to make a choice among different products given income

        \end{enumerate}

      \item \textbf{Trade-off}:

        \begin{enumerate}

          \item A trade-off occurs when people forfeit one activity to do another activity or when people give up more of one activity to get more of another activity

          \item Ex. The U.S. Government reduces its spending on education and social welfare to raise its defense spending

        \end{enumerate}

      \item \textbf{Opportunity Cost}:

        \begin{enumerate}

          \item The opportunity cost is the value of the second-best option given up to choose the best option

          \item It is the quantitative measure of trade-off relationships among alternative options

          \item Economists often express opportunity cost as a dollar value to compare alternatives

          \item Not all opportunity costs are measured in dollar or monetary values

          \item Ex. Toyota can produce 2 Corollas or 1 Camry with the same resources. What is the opportunity cost of 1 Camry? 2 Corollas

        \end{enumerate}

      \item \textbf{Efficiency}:

        \begin{enumerate}

          \item Efficiency occurs when resources are used to create the greatest economic value within a society

          \item Productive efficiency occurs when goods or services are produced at the minimum cost or when goods or services are produced to maximize profit

          \item Allocative efficiency occurs when goods or services produced satisfy consumers' preferences most or when consumers obtain goods or services to maximize their efficiency

        \end{enumerate}

      \item \textbf{Economic Growth}:

        \begin{enumerate}

          \item Economic growth is an increase in the production of goods or services

          \item Three ways to obtain economic growth exist:

            \begin{enumerate}

              \item Increase resources (labor, capital, natural resources, and entrepreneurship)

              \item Technological advancement (progress) or positive technological change or innovation

              \item Specialization and trade — Gains from trade with the right terms of trade

            \end{enumerate}

        \end{enumerate}

      \item \textbf{Economic Model}

        \begin{enumerate}

          \item A simplified representation or version of a real economic phenomenon (situation)

          \item Focuses on essentials of the complex reality by simplifying through assumptions, getting useful and approximate answers to economic problems, and obtaining predictions for the future

          \item May be expressed in three ways: words, graphs, and equations

          \item Ex. Circular flow model (words and diagrams) and Production possibilites frontier (in words, graphs, and equations)

        \end{enumerate}

      \item Characteristics of a good model

        \begin{enumerate}

          \item Makes clear and reasonable assumptions (based on economic theory and data)

          \item Predicts cause and effect relationship among economic variables (focuses on causal relationship between two economic variables, assuming all else to be constant)

          \item Accurately describes the real world (data)

            \begin{enumerate}

              \item Does not need to include all complex details from reality

              \item Needs to approximately replicate reality

            \end{enumerate}

        \end{enumerate}

      \item Developing a good economic model

        \begin{enumerate}

          \item Make reasonable assumptions based on theory

          \item Formulate a testable hypothesis\footnote{A statement about causal relationships between two economic variables that may be true or not}

          \item Test a hypothesis through data collection and statistical models: null hypothesis (the hypothesis one wants to accept) and alternative hypothesis (hypothesis one wants to reject)

          \item Revise the model by changing assumptions, data, or both if the null hypothesis is rejected

          \item Use the revised model to explain or predict economic events

        \end{enumerate}

      \item Developing a demand model

        \begin{enumerate}

          \item Assume that everything else is constant, other than the price of a specific product

          \item Theoretical model: $P=a+bQ_d$, where $P$ is price, and $Q_d$ is the quantity determined

          \item Null hypothesis ($H_0$): There is a negative relationship between price and quantity determined ($b<0$)

          \item Alternative hypothesis ($H_A$): There is either no relationship or a positive relationship between the price and quantity determined ($b\geq0$)

          \item Statistical model: $P=a+bQ_d+\varepsilon$, where $\varepsilon$ is the error term

          \item Collect data on price and quantity demanded to estimate $b$

          \item If data rejects the null hypothesis, revise the model by changing data or assumptions. Collect data once again on price and quantity demanded to estimate $b$

          \item Ex. $P=10-2Q_d$ — If the quantity demanded increases by 1, the price will decline by $\$ 2$, or, if the price rises by $\$ 1$, the quantity demanded will decline by $.5$

        \end{enumerate}

      \item Economic Model Examples

        \begin{enumerate}

          \item Basic Circular Flow Model

            \begin{enumerate}

              \item Assumptions

                \begin{enumerate}

                  \item Two economic agents: households and firms

                  \item Two markets: product market and factor market

                \end{enumerate}

              \item One of the most basic models that shows the flow of money, goods and services, and inputs

              \item In the product market, households spend money to buy goods and services that firms obtain revenue by making and selling

              \item In the factor market, households earn income (wage, rent, and profit) by providing inputs that firms pay for (cost) to produce goods and services

            \end{enumerate}

            \begin{center}
              \begin{figure}[h]
                \centering
                \tikzset{every picture/.style={line width=0.75pt}} %set default line width to 0.75pt        

\begin{tikzpicture}[x=0.75pt,y=0.75pt,yscale=-1,xscale=1]
%uncomment if require: \path (0,300); %set diagram left start at 0, and has height of 300

%Rounded Rect [id:dp3465100630073805] 
\draw  [fill={rgb, 255:red, 74; green, 144; blue, 226 }  ,fill opacity=1 ] (263,22.2) .. controls (263,16.57) and (267.57,12) .. (273.2,12) -- (389.8,12) .. controls (395.43,12) and (400,16.57) .. (400,22.2) -- (400,52.8) .. controls (400,58.43) and (395.43,63) .. (389.8,63) -- (273.2,63) .. controls (267.57,63) and (263,58.43) .. (263,52.8) -- cycle ;
%Rounded Rect [id:dp7214464129129555] 
\draw  [fill={rgb, 255:red, 255; green, 255; blue, 255 }  ,fill opacity=1 ] (273.5,25.5) .. controls (273.5,21.08) and (277.08,17.5) .. (281.5,17.5) -- (381.5,17.5) .. controls (385.92,17.5) and (389.5,21.08) .. (389.5,25.5) -- (389.5,49.5) .. controls (389.5,53.92) and (385.92,57.5) .. (381.5,57.5) -- (281.5,57.5) .. controls (277.08,57.5) and (273.5,53.92) .. (273.5,49.5) -- cycle ;
%Rounded Rect [id:dp8133849265641938] 
\draw  [fill={rgb, 255:red, 74; green, 144; blue, 226 }  ,fill opacity=1 ] (458,124.2) .. controls (458,118.57) and (462.57,114) .. (468.2,114) -- (584.8,114) .. controls (590.43,114) and (595,118.57) .. (595,124.2) -- (595,154.8) .. controls (595,160.43) and (590.43,165) .. (584.8,165) -- (468.2,165) .. controls (462.57,165) and (458,160.43) .. (458,154.8) -- cycle ;
%Rounded Rect [id:dp30443809253463616] 
\draw  [fill={rgb, 255:red, 255; green, 255; blue, 255 }  ,fill opacity=1 ] (468.5,127.5) .. controls (468.5,123.08) and (472.08,119.5) .. (476.5,119.5) -- (576.5,119.5) .. controls (580.92,119.5) and (584.5,123.08) .. (584.5,127.5) -- (584.5,151.5) .. controls (584.5,155.92) and (580.92,159.5) .. (576.5,159.5) -- (476.5,159.5) .. controls (472.08,159.5) and (468.5,155.92) .. (468.5,151.5) -- cycle ;
%Rounded Rect [id:dp5907574749667166] 
\draw  [fill={rgb, 255:red, 74; green, 144; blue, 226 }  ,fill opacity=1 ] (89,125.2) .. controls (89,119.57) and (93.57,115) .. (99.2,115) -- (215.8,115) .. controls (221.43,115) and (226,119.57) .. (226,125.2) -- (226,155.8) .. controls (226,161.43) and (221.43,166) .. (215.8,166) -- (99.2,166) .. controls (93.57,166) and (89,161.43) .. (89,155.8) -- cycle ;
%Rounded Rect [id:dp5310892985635383] 
\draw  [fill={rgb, 255:red, 255; green, 255; blue, 255 }  ,fill opacity=1 ] (99.5,128.5) .. controls (99.5,124.08) and (103.08,120.5) .. (107.5,120.5) -- (207.5,120.5) .. controls (211.92,120.5) and (215.5,124.08) .. (215.5,128.5) -- (215.5,152.5) .. controls (215.5,156.92) and (211.92,160.5) .. (207.5,160.5) -- (107.5,160.5) .. controls (103.08,160.5) and (99.5,156.92) .. (99.5,152.5) -- cycle ;
%Rounded Rect [id:dp827002824816521] 
\draw  [fill={rgb, 255:red, 74; green, 144; blue, 226 }  ,fill opacity=1 ] (263,237.2) .. controls (263,231.57) and (267.57,227) .. (273.2,227) -- (389.8,227) .. controls (395.43,227) and (400,231.57) .. (400,237.2) -- (400,267.8) .. controls (400,273.43) and (395.43,278) .. (389.8,278) -- (273.2,278) .. controls (267.57,278) and (263,273.43) .. (263,267.8) -- cycle ;
%Rounded Rect [id:dp4820855032011546] 
\draw  [fill={rgb, 255:red, 255; green, 255; blue, 255 }  ,fill opacity=1 ] (273.5,240.5) .. controls (273.5,236.08) and (277.08,232.5) .. (281.5,232.5) -- (381.5,232.5) .. controls (385.92,232.5) and (389.5,236.08) .. (389.5,240.5) -- (389.5,264.5) .. controls (389.5,268.92) and (385.92,272.5) .. (381.5,272.5) -- (281.5,272.5) .. controls (277.08,272.5) and (273.5,268.92) .. (273.5,264.5) -- cycle ;
%Straight Lines [id:da22766006944212214] 
\draw [color={rgb, 255:red, 208; green, 2; blue, 27 }  ,draw opacity=1 ][fill={rgb, 255:red, 208; green, 2; blue, 27 }  ,fill opacity=1 ][line width=2.25]    (259.98,56.78) -- (215.8,115) ;
\draw [shift={(263,52.8)}, rotate = 127.19] [fill={rgb, 255:red, 208; green, 2; blue, 27 }  ,fill opacity=1 ][line width=0.08]  [draw opacity=0] (16.07,-7.72) -- (0,0) -- (16.07,7.72) -- (10.67,0) -- cycle    ;
%Straight Lines [id:da10983822042305391] 
\draw [color={rgb, 255:red, 126; green, 211; blue, 33 }  ,draw opacity=1 ][fill={rgb, 255:red, 126; green, 211; blue, 33 }  ,fill opacity=1 ][line width=2.25]    (263,22.2) -- (103.55,112.54) ;
\draw [shift={(99.2,115)}, rotate = 330.47] [fill={rgb, 255:red, 126; green, 211; blue, 33 }  ,fill opacity=1 ][line width=0.08]  [draw opacity=0] (16.07,-7.72) -- (0,0) -- (16.07,7.72) -- (10.67,0) -- cycle    ;
%Straight Lines [id:da13493886382595943] 
\draw [color={rgb, 255:red, 126; green, 211; blue, 33 }  ,draw opacity=1 ][fill={rgb, 255:red, 126; green, 211; blue, 33 }  ,fill opacity=1 ][line width=2.25]    (99.2,166) -- (258.75,265.16) ;
\draw [shift={(263,267.8)}, rotate = 211.86] [fill={rgb, 255:red, 126; green, 211; blue, 33 }  ,fill opacity=1 ][line width=0.08]  [draw opacity=0] (16.07,-7.72) -- (0,0) -- (16.07,7.72) -- (10.67,0) -- cycle    ;
%Straight Lines [id:da7517724119578088] 
\draw [color={rgb, 255:red, 208; green, 2; blue, 27 }  ,draw opacity=1 ][fill={rgb, 255:red, 208; green, 2; blue, 27 }  ,fill opacity=1 ][line width=2.25]    (219.23,169.64) -- (273.2,227) ;
\draw [shift={(215.8,166)}, rotate = 46.74] [fill={rgb, 255:red, 208; green, 2; blue, 27 }  ,fill opacity=1 ][line width=0.08]  [draw opacity=0] (16.07,-7.72) -- (0,0) -- (16.07,7.72) -- (10.67,0) -- cycle    ;
%Straight Lines [id:da36357253787164334] 
\draw [color={rgb, 255:red, 208; green, 2; blue, 27 }  ,draw opacity=1 ][fill={rgb, 255:red, 208; green, 2; blue, 27 }  ,fill opacity=1 ][line width=2.25]    (393.72,223.9) -- (468.2,165) ;
\draw [shift={(389.8,227)}, rotate = 321.66] [fill={rgb, 255:red, 208; green, 2; blue, 27 }  ,fill opacity=1 ][line width=0.08]  [draw opacity=0] (16.07,-7.72) -- (0,0) -- (16.07,7.72) -- (10.67,0) -- cycle    ;
%Straight Lines [id:da17368072688608072] 
\draw [color={rgb, 255:red, 208; green, 2; blue, 27 }  ,draw opacity=1 ][fill={rgb, 255:red, 208; green, 2; blue, 27 }  ,fill opacity=1 ][line width=2.25]    (400,52.8) -- (464.48,110.66) ;
\draw [shift={(468.2,114)}, rotate = 221.9] [fill={rgb, 255:red, 208; green, 2; blue, 27 }  ,fill opacity=1 ][line width=0.08]  [draw opacity=0] (16.07,-7.72) -- (0,0) -- (16.07,7.72) -- (10.67,0) -- cycle    ;
%Straight Lines [id:da6621718953660283] 
\draw [color={rgb, 255:red, 126; green, 211; blue, 33 }  ,draw opacity=1 ][fill={rgb, 255:red, 126; green, 211; blue, 33 }  ,fill opacity=1 ][line width=2.25]    (400,267.8) -- (580.43,167.43) ;
\draw [shift={(584.8,165)}, rotate = 510.91] [fill={rgb, 255:red, 126; green, 211; blue, 33 }  ,fill opacity=1 ][line width=0.08]  [draw opacity=0] (16.07,-7.72) -- (0,0) -- (16.07,7.72) -- (10.67,0) -- cycle    ;
%Straight Lines [id:da4548385579774181] 
\draw [color={rgb, 255:red, 126; green, 211; blue, 33 }  ,draw opacity=1 ][fill={rgb, 255:red, 126; green, 211; blue, 33 }  ,fill opacity=1 ][line width=2.25]    (404.48,24.42) -- (584.8,114) ;
\draw [shift={(400,22.2)}, rotate = 26.42] [fill={rgb, 255:red, 126; green, 211; blue, 33 }  ,fill opacity=1 ][line width=0.08]  [draw opacity=0] (16.07,-7.72) -- (0,0) -- (16.07,7.72) -- (10.67,0) -- cycle    ;
%Straight Lines [id:da014368688940361807] 
\draw [draw opacity=0]   (157.5,140.5) -- (242,141) ;
%Straight Lines [id:da27176174044752055] 
\draw [draw opacity=0]   (526.5,139.5) -- (437,139.5) ;

% Text Node
\draw (331.5,37.5) node   [align=left] {Households};
% Text Node
\draw (157.5,140.5) node   [align=left] {Product Market};
% Text Node
\draw (526.5,139.5) node   [align=left] {Factor Market};
% Text Node
\draw (331.5,252.5) node   [align=left] {Firms};
% Text Node
\draw (244,141) node [anchor=west] [inner sep=0.75pt]   [align=left] {Goods or\\Services};
% Text Node
\draw (435,139.5) node [anchor=east] [inner sep=0.75pt]   [align=left] {Inputs};
% Text Node
\draw (494.4,219.4) node [anchor=north west][inner sep=0.75pt]   [align=left] {Cost};
% Text Node
\draw (494.4,65.1) node [anchor=south west] [inner sep=0.75pt]   [align=left] {\begin{minipage}[lt]{92.5pt}\setlength\topsep{0pt}
\begin{center}
Income\\(Wage, Rent, Profit)
\end{center}

\end{minipage}};
% Text Node
\draw (179.1,219.9) node [anchor=north east] [inner sep=0.75pt]   [align=left] {Revenue};
% Text Node
\draw (179.1,65.6) node [anchor=south east] [inner sep=0.75pt]   [align=left] {Spending};


\end{tikzpicture}


                \caption{Example Basic Circular Flow Model}
                \label{fig:1}
              \end{figure}
            \end{center}

            \begin{enumerate}

              \item Missing Components

                \begin{enumerate}

                  \item Missing economic agents: Governments

                  \item Missing markets: financial and international market

                  \item A modified circular flow model includes these components

                \end{enumerate}

            \end{enumerate}

          \item Production Possibilities Frontier (PPF)

            \begin{enumerate}

              \item A production possibilities frontier is a curve that shows the maximum attainable (feasible) combinations of two products (goods or services) that can be produced given resources and technology

              \item Assumptions:

                \begin{enumerate}

                  \item There are only two products (goods or services)

                  \item Resources and technology are fixed

                \end{enumerate}

              \item This model can illustrate the aforementioned five basic economic concepts

              \item Two types of PPF exist:

                \begin{enumerate}

                  \item Linear — Assumes all resources have the same efficiency in producing two products. Opportunity cost is constant at any point. The slope of PPF is equal to the marginal opportunity cost. Unrealistic but can be used for simplicity.

                    \begin{itemize}

                      \item Scarcity — Any point on or below the line is feasible, while any point above the line is infeasible (due to lack of resources)

                      \item Trade-offs — Occur when an increase in the production of one product cause a reduction in the production of another product. A trade-off always occurs when moving from one efficient point to another, but only sometimes occur when moving from inefficient to inefficient or inefficient to efficient

                      \item Efficiency — An inefficient point is feasible, but below the line. An efficient point is anywhere along the line. An allocatively efficient point is any feasible point that satisfies consumer preferences or maximizes customer utility. Such points may not be represented on a PPF without consumer preferences

                      \item Opportunity Cost — The opportunity cost is measured by the number of the other good given up to choose the best option because there are only two options present on a PPF. The total opportunity cost is measured by the number of the other good given up to produce more of one good. Marginal opportunity cost is measured by the number of the other good given up to produce one more of one good (i.e. the slope)

                      \item Economic Growth — Economic growth can be demonstrated by an increase in the production of goods or services (i.e. a shift to the right of a PPF). Economic growth can occur for the aforementioned three reasons.

                        \begin{itemize}

                          \item More resources — A parallel shift to the right occurs

                          \item Technological advancement — If the advancement is in the good on the Y-axis, the slope increases. If the advancement is in the good on the X-axis, the slope decreases. These are pivotal shifts. If the advancement occurs equally with respect to both goods, a parallel, rightwards shift occurs
                            
                        \end{itemize}

                    \end{itemize}

                  \item Concave — Assumes that each resource has different efficiency in production. The marginal opportunity cost (slope) is increasing. More (though not completely) realistic than the linear counterpart

                \end{enumerate}

            \end{enumerate}

        \end{enumerate}

      \item Positive vs. Normative

        \begin{enumerate}

          \item Positive Analysis

            \begin{enumerate}

              \item A factual claim about causal relationships between two economic variables (hypothesis)

              \item It is testable through data and statistical model(s)

              \item Ex. Higher minimum wage will raise unemployment

            \end{enumerate}

          \item Normative Analysis

            \begin{enumerate}

              \item A claim on how the world should be based on beliefs, political views, and values

              \item It is not testable through data and statistical model(s)

              \item A positive statement may help to form a normative statement

              \item Ex. Minimum wage should be raised to raise the income of the low-skilled (educated) workers

            \end{enumerate}

        \end{enumerate}

      \item Microeconomics vs. Macroeconomics

        \begin{enumerate}

          \item Microeconomics

            \begin{enumerate}

              \item A study of how households and firms make decisions, how they interact in the market, and how the government influences their decisions

              \item Microeconomic variables include the prices, quantity demanded, and quantity supplied of a specific product

              \item Microeconomic model includes demand and supply model for a specific product

              \item Microeconomic policies include price control, excise tax, trade policies, anti-trust laws, price regulation, and environmental policies

            \end{enumerate}

          \item Macroeconomics

            \begin{enumerate}

              \item A study of the economy as a whole and how policy-makers manage the growth and behavior of the overall economy

              \item Macroeconomic variables include total production (GDP), price level, inflation, economic growth rate, and unemployment

              \item Macroeconomic model includes aggregate demand and aggregate supply

              \item Macroeconomic topics include business cycle and economic growth

              \item Macroeconomic policies include fiscal and monetary policies

            \end{enumerate}

        \end{enumerate}

      \item What is a Market?

        \begin{enumerate}

          \item A market is a place where a group of buyers and sellers come together to trade any specific product (good, service, input, or financial asset)

          \item Always two sides of the market:

            \begin{enumerate}

              \item The demand represents the buyers' side

              \item The supply represents the sellers' side

            \end{enumerate}

          \item In the market, buyers and sellers interact with each other and determine price and quantity for a specific product

          \item As a result, the demand and supply model is used in analyzing the market

        \end{enumerate}

      \item Types of Markets:

        \begin{enumerate}

          \item Location-Based

            \begin{enumerate}

              \item Physical Markets

                \begin{enumerate}

                 \item Farmers' Market

                 \item Shopping Mall (e.g. Sun Valley Mall)

                 \item Book Stores (e.g. Barnes \& Noble)

                 \item Grocery Stores (e.g. Safeway, Walmart, Costco)

               \end{enumerate}

             \item Virtual Markets

               \begin{enumerate}

                 \item eBay, Amazon, Kindle

               \end{enumerate}

            \end{enumerate}

          \item Product-Based

            \begin{enumerate}

              \item Goods or services are traded in the product market

              \item Goods are tangible products such as cars, books, notes, gold, etc.

              \item Services are intangible products such as financial services, airline services, education, etc.

              \item Households are buyers of goods or services, and firms are sellers of goods or services

              \item Households spend money to purchase goods or services, but firms earn a profit by selling goods or services

            \end{enumerate}

          \item Factor-Based

            \begin{enumerate}

              \item Inputs are traded in the factor market

              \item Labor includes full-time, part-time, regular, or temporary workers, and is measured by working hours

              \item Capital is a product itself but also used as inputs and includes machines, tools, and buildings

              \item Natural resources include land, oil, water, air, etc

              \item Entrepreneurships are skills, abilities, and knowledge of managers

              \item Firms are buyers but households are sellers of inputs

            \end{enumerate}

          \item Financial-Based

            \begin{enumerate}

              \item Financial assets are traded in the financial market

              \item Financial assets include checking and saving accounts, stocks, bonds, mutual funds, insurance, etc.

              \item Borrowers of funds are buyers of financial assets, while savers or lenders are sellers of financial assets

              \item This market is not included in the basic circular flow model, but will be in the modified one

              \item Financial assets (capital) is not included in inputs, but physical capital is included in inputs

            \end{enumerate}

          \item Demand and supply model is used to analyze markets

          \item Demand shows the relationship between price and quantity demanded of a specific product

          \item Demand model can be expressed in words (law of demand), in a table (demand schedule), or in a graph (demand curve)

        \end{enumerate}

      \item The Law of Demand

        \begin{enumerate}

          \item This law describes the relationship between price and quantity demanded of a specific product

          \item There is an inverse (negative) relationship between price and quantity demanded of a product, ceteris paribus\footnote{Holding all else constant (in latin)}

          \item The quantity demanded ($Q_d$) is the amount of a product that buyers are willing and able to buy

          \item Negative relationship means that, with a higher price, the quantity demanded will decline, and vice versa

        \end{enumerate}

      \item Why does the law of demand hold?

        \begin{enumerate}

          \item Increasing marginal opportunity cost — With higher price, a product will become more expensive than other substitutes and consumers buy less of it

          \item Diminishing marginal utility — With higher quantity demanded, consumers will pay less for additional units due to lower marginal utility
            
          \item Exceptiion: A Giffen good is a good that has a positive relationship between price and quantity demanded, which violates this law

        \end{enumerate}

      \item Demand Schedule

        \begin{enumerate}

          \item An individual demand schedule is a table that shows the relationship between price and individual quantity demanded of a product, ceteris paribus

          \item Individual quantity demanded is a quantity demanded of a product for only one consumer

          \item Market demand schedule is a table that shows the relationship between price and the market quantity demanded of a product, ceteris paribus

          \item Market quantity demanded is the sum of individual quantity demanded or the quantity demanded of a product for the whole market

        \end{enumerate}

      \item Demand Curves

        \begin{enumerate}

          \item Individual demand curves are graphs or figures that show the relationship between price and individual quantity demanded of a product, ceteris paribus

          \item Market demand curves are graphs or figures that show the relationship between price and market quantity demanded of a product, ceteris paribus

          \item Market demand curves are the horizontal sums of individual demand curves at a given price

          \item Market demand curves are assumed to be linear because they are mostly linear for buyers

        \end{enumerate}

      \item Non-price factors that shift demand

        \begin{enumerate}

          \item There are five major factors other than the price that influence the demand of a product:

            \begin{enumerate}

              \item Income

              \item Preference (Taste)

              \item Prices of substitutes

              \item Number of buyers

              \item Expected price of a product

            \end{enumerate}

        \end{enumerate}

      \item How each factor influences demand

        \begin{enumerate}

          \item Income

            \begin{enumerate}

              \item Consumers' income will affect the demand of a product in different ways depending on the type of product

              \item A product is a normal good if increase in income of consumers will increase the demand of that product and vice versa

              \item There is a positive relationship between income and demand, ceteris paribus

              \item Most goods are normal (e.g. New cars, lamps, laptops, etc.)

              \item A product is an inferior good if increase in income of a consumer will decrease the demand, and vice versa

              \item There is a negative relationship between income and demand, ceteris paribus

              \item Few goods are inferior (e.g. Used cars, instant noodles, fast food, etc.)

            \end{enumerate}

          \item Preference (Taste)

            \begin{enumerate}

              \item Change in preference, or the like or dislike towards a product, affects the demand in various ways

              \item In economics, we only care about how peoples' preferences affect their consumption behaviors, whereas sociology may care about how preferences change

              \item There are a variety of factors that influence consumers' preferences — Health concerns, advertising, outbreak of disease, research results, etc.

              \item Ex. High obesity rates lower the demand for high fat products, but increase the demand for low fat products

            \end{enumerate}

          \item Prices of substitutes

            \begin{enumerate}

              \item Substitutes are products that can replace each other because of similar usages

              \item The rise in the price of a substitute will increase the demand for a product, and vice versa

              \item There is a positive relationship between the price of a substitute and demand for a product

              \item Ex. Competition between Iphones, Samsung Galaxy, and Pinephones

              \item Complements are products that can be consumed together

              \item The rise in the price of a complement will decrease the demand for a product, and vice versa

              \item There is a negative relationship between the price of a complement and demand for the product

              \item Ex. Peanut Butter and Jelly

              \item Unrelated products have no relationship at all

              \item Ex. Laptop and Refrigerator

            \end{enumerate}
  
          \item Number of buyers

            \begin{enumerate}

              \item An increase in the number of buyers will increase the (market) demand for a product

              \item There is a positive relationship between number of buyers and demand, ceteris paribus

              \item There are a variety of factors that influence number of buyers:

                \begin{enumerate}

                  \item Birth and Death rates

                  \item Migration among regions

                  \item Immigration among countries

                  \item Change in demographics

                \end{enumerate}

              \item Ex. Increase in birth rate would result in an increase in the demand for birthing centers

            \end{enumerate}

          \item Expected (future) price of a product

            \begin{enumerate}

              \item If the price of a product is expected to rise in the future, the current demand for a product will rise, but the future demand will fall, and vice versa

              \item There is a positive relationship between the expected price and current demand for a product, ceteris paribus

              \item Ex. Expected rise in an airline ticket price during summer will increase the demand during spring

            \end{enumerate}

        \end{enumerate}

      \item Change in quantity demanded ($Q_d$) vs. demand ($D$)

        \begin{enumerate}
            
          \item Change in $Q_d$ occurs when $Q_d$ changes as a result of change in the price of a product, ceteris paribus

          \item Change in $Q_d$ shows the movement along the demand curve or the movement from one point to another on the demand curve

          \item Decrease in $Q_d$ occurs as a result of rise in the price of a product, ceteris paribus, which can be shown as the upward movement along the demand curve

          \item Increase in $Q_d$ occurs as a result of decrease in the price of a prodcut, ceteris paribus, which can be shown as the downward movement along the demand curve

          \item Change in $D$ occurs when $Q_d$ changes as a result pf change in one of the non-price factors

          \item Change in $D$ shows the shift in the demand curve or the change in $Q_d$ at each given price

          \item Decrease in demand occurs when $Q_d$ decrease at each given price, which can be shown as shifting demand to the left

          \item Increase in demand occurs when $Q_d$ increases at each given price, which can be shown as shifting demand to the right

        \end{enumerate}

      \item Supply

        \begin{enumerate}

          \item Supply represents the sellers' side

          \item Supply shows the relationship between price and quantity supplied of a specific product

          \item Supply model can be expressed in words (Law of Supply), in a table (supply schedule), or in a graph (supply curve)

        \end{enumerate}

      \item Law of Supply

        \begin{enumerate}

          \item A law that describes the relationship between price and quantity supplied of a specific product

          \item There is a positive relationship between price and quantity supplied of a product, ceteris paribus

          \item The quantity supplied ($Q_s$) is the amound of a product that sellers are willing and able to provide

        \end{enumerate}

      \item Why does the law of supply hold?

        \begin{enumerate}

          \item Increasing profit — With higher price, a product will become more profitable than other products and sellers offer more of it, and vice versa

          \item Increasing marginal opportunity cost (marginal cost) — With higher quantity supplied of a product, producers will cost more to produce additional units and charge more for it, and vice versa

        \end{enumerate}

      \item Supply Schedule

        \begin{enumerate}

          \item An individual supply schedule is a table that shows the relationship between price and individual quantity supplied of a product, ceteris paribus

          \item Individual quantity supplied is a quantity supplied of a product for only one seller

          \item Market supply schedule is a table that shows the relationship between price and the market quantity supplied of a product, ceteris paribus

          \item Market quantity supplied is the sum of individual quantity supplied or the quantity supplied of a product for the whole market

        \end{enumerate}

      \item Supply Curve

        \begin{enumerate}

          \item Individual supply curves are graphs or figures that show the relationship between price and individual quantity supplied of a product, ceteris paribus

          \item Market supply curves are graphs or figures that show the relationship between price and market quantity supplied of a product, ceteris paribus

          \item Market supply curves are the horizontal sums of individual supply curves at a given price

          \item Market supply curves are assumed to be linear because they are approximately linear for many sellers, even though few sellers are not

        \end{enumerate}

      \item Non-price factors that shift supply

        \begin{enumerate}

          \item Technology

          \item Prices of inputs

          \item Prices of other products

          \item Number of sellers

          \item Expected price of a product

        \end{enumerate}

      \item How each factor influences supply

        \begin{enumerate}

          \item Technology

            \begin{enumerate}

              \item With positive technological change (progress), more may be supplied

              \item Sellers can make more of a product (higher quantity supplied) given resources

              \item There is a positive relationship between innovation and supply, ceteris paribus

              \item Ex. More efficient assembly of a Pinephone

              \item With negative technological change, less of a quantity may be supplied

              \item There is a negative relationship between negative technological change and supply, ceteris paribus

              \item Ex. Use of deprecated equipment or tools

            \end{enumerate}

          \item Prices of inputs

            \begin{enumerate}

              \item The rise in the price of an input will lower the supply of a product due to higher cost of production, and vice versa

              \item There is a negative relationship between price of inputs and supply, ceteris paribus

              \item Ex. Silicon shortage for Pine64

            \end{enumerate}

          \item Prices of substitutes

            \begin{enumerate}

              \item Substitutes are products that can replace one another by a seller

              \item The rise in the price of a substitute will decrease the supply of a product and vice versa

              \item There is a negative relationship between price of a substitute and supply for a product

              \item Ex. Tablet and Laptop

              \item Complements are products that can be produced together by a seller

              \item The rise in the price of a complement will increase the supply of a product and vice versa

              \item There is a positive relationship between price of a complement and supply for a product

              \item Ex. Oil and Natural Gas

            \end{enumerate}

          \item Number of sellers

            \begin{enumerate}
                
              \item Increase in the number of sellers will increase the (market) supply for a product because the market quantity supplied of a product will rise, and vice versa

              \item There is a positive relationship between number of sellers and supply, ceteris paribus

              \item There are several factors that influence the number of sellers — Profitability and barriers to entry and exit

                \begin{enumerate}

                  \item Positive profit will attract new firms to enter the market and increase the supply of a product

                  \item Negative profit (loss) will make some existing firms leave the market and lower the supply of a product

                  \item High degree of barriers to entry or exit will make it harder to change the number of sellers, whereas low degree of barriers to entry or exit will make it easier to change the number of sellers

                  \item Perfectly competitive markets and monopolistically competitive markets have no barriers to entry and exit, but a monopoly market has the strongest barriers, while an oligopoly market has some degree of barriers

                \end{enumerate}

              \item Ex. Barriers to entry — Tax benefits, patents, strong brands, etc.
              
              \item Ex. Barriers to exit — Asset write-offs, closure costs, etc.

            \end{enumerate}

          \item Expected price of a product

            \begin{enumerate}

              \item The rise in the expected price of a product will lower the current supply of that product, but increase the future supply of a product, ceteris paribus, and vice versa

              \item There is a negative relationship between the expected price and the current supply, ceteris paribus

              \item The rise in the price of an airline ticket during the summer will decrease the supply for an airline ticket during spring

            \end{enumerate}

        \end{enumerate}

      \item Change in quantity supplied ($Q_s$) vs. supply ($S$)

        \begin{enumerate}

          \item Change in $Q_s$ occurs when $Q_s$ changes as a result of a change in the price of a product, ceteris paribus

          \item Change in $Q_s$ shows the movement along the supply curve or the movement from one point to another on the supply curve

          \item Decrease in $Q_s$ occurs as a result of a decrease in the price of a product, ceteris paribus, which can be shown as the downward movement along the supply curve

          \item Increase in $Q_s$ occurs as a result of an increase in the price of a product, which can be shown as the upward movement along the supply curve

          \item Change in $S$ occurs when $Q_s$ changes as a result of a change in one of the non-price factors, ceteris paribus

          \item Change in $S$ shows the shift in the supply curve or the change in $Q_s$ at each given price

          \item Decrease in supply occurs when $Q_s$ decreases at each given price, which can be shown as shifting the supply to the left

          \item Increase in supply occurs when $Q_s$ increases at each given price, which can be shown as shifting the supply to the right

        \end{enumerate}

      \item Market Disequilibrium

        \begin{enumerate}

          \item A situation where there is an incentive to change (deviate) in the market

          \item There are two cases of disequilibrium: surplus and shortage

          \item Surplus (excess supply) is a situation where the quantity supplied ($Q_s$) is greater than the quantity demanded ($Q_d$) at a price greater than equilibrium price

            \begin{enumerate}

              \item Sellers want to sell more than buyers want to purchase, and has an incentive to lower the price to raise their revenue

            \end{enumerate}

          \item Shortage (excess demand) is a situation where the quantity supplied ($Q_s$) is less than the quantity demanded ($Q_d$) at a price less than equilibrium price

            \begin{enumerate}

              \item Buyers want to buy more than sellers want to sell, and has an incentive to raise the price to increase their revenue

            \end{enumerate}

          \item With a surplus, sellers want to lower the price to equilibrium price, whereas, with shortages, sellers want to raise the price to equilibrium price

          \item Eventually, the market transitions from disequilibrium to equilibrium by changing the price, which is the “Invisible hand” referenced by Adam Smith

        \end{enumerate}
        
      \item Market Equilibrium

        \begin{enumerate}

          \item A situation in which there is no incentive for deviation in the market

          \item At market equilibrium:

            \begin{enumerate}

              \item Demand curve intersects supply curve

              \item Quantity supplied ($Q_s$) = Quantity demanded ($Q_d$)

              \item Every seller who wants to provide a product finds buyers who want to purchase it at the equilibrium price (known as Market Clearing)

              \item There is neither a surplus nor shortage

            \end{enumerate}

        \end{enumerate}
    
      \item Change in Market Equilibrium

        \begin{enumerate}

          \item Change in the market conditions occurs when non-price factors that shift demand or supply change

          \item There are three different cases of the market condition changes

            \begin{enumerate}

              \item Shift in only demand

              \item Shift in only supply

              \item Shift in both demand and supply

            \end{enumerate}

          \item Shift in only demand

            \begin{enumerate}

              \item Increase in demand or shifting the demand to the right will increase both the equilibrium price and quantity

              \item Decrease in demand of shifting the demand to the left will lower both the equilibrium price and quantity

            \end{enumerate}

          \item Shift in only supply

            \begin{enumerate}

              \item Increase in supply or shifting the supply to the right will decrease the equilibrium price but increase the equilibrium quantity

              \item Decrease in supply or shifting the supply to the left will increase the equilibrium price but decrease the equilibrium quantity

            \end{enumerate}
            
          \item Shift in both supply and demand

            \begin{enumerate}

              \item Increase in both supply and demand will increase the equilibrium quantity, but may increase, decrease, or keep the equilibrium price

              \item Decrease in both supply and demand will decrease the equilibrium quantity, but may increase, decrease, or keep the equilibrium price

              \item Increase in supply and decrease in demand will decrease the equilibrium price, but may increase, decrease, or keep the equilibrium quantity

              \item Decrease in supply and increase in demand will increase the equilibrium price, but may increase, decrease, or keep the equilibrium quantity

            \end{enumerate}

        \end{enumerate}

    \end{enumerate}

\end{document}

