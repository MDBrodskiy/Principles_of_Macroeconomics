%%%%%%%%%%%%%%%%%%%%%%%%%%%%%%%%%%%%%%%%%%%%%%%%%%%%%%%%%%%%%%%%%%%%%%%%%%%%%%%%%%%%%%%%%%%%%%%%%%%%%%%%%%%%%%%%%%%%%%%%%%%%%%%%%%%%%%%%%%%%%%%%%%%%%%%%%%%%%%%%%%%%%%%%%%%%%%%%%%%%%%%%%%%%
% Written By Michael Brodskiy
% Class: Principles of Macroeconomics
% Professor: H. Yoon
%%%%%%%%%%%%%%%%%%%%%%%%%%%%%%%%%%%%%%%%%%%%%%%%%%%%%%%%%%%%%%%%%%%%%%%%%%%%%%%%%%%%%%%%%%%%%%%%%%%%%%%%%%%%%%%%%%%%%%%%%%%%%%%%%%%%%%%%%%%%%%%%%%%%%%%%%%%%%%%%%%%%%%%%%%%%%%%%%%%%%%%%%%%%

\documentclass[12pt]{article} 
\usepackage{alphalph}
\usepackage[utf8]{inputenc}
\usepackage[russian,english]{babel}
\usepackage{titling}
\usepackage{amsmath}
\usepackage{graphicx}
\usepackage{enumitem}
\usepackage{amssymb}
\usepackage[super]{nth}
\usepackage{everysel}
\usepackage{ragged2e}
\usepackage{geometry}
\usepackage{fancyhdr}
\usepackage{cancel}
\usepackage{siunitx}
\usepackage{xcolor}
\usepackage{physics}
\usepackage{tikz}
\usepackage{mathdots}
\usepackage{yhmath}
\usepackage{color}
\usepackage{array}
\usepackage{multirow}
\usepackage{gensymb}
\usepackage{tabularx}
\usepackage{extarrows}
\usepackage{booktabs}
\usetikzlibrary{fadings}
\usetikzlibrary{patterns}
\usetikzlibrary{shadows.blur}
\usetikzlibrary{shapes}


\geometry{top=1.0in,bottom=1.0in,left=1.0in,right=1.0in}
\newcommand{\subtitle}[1]{%
  \posttitle{%
    \par\end{center}
    \begin{center}\large#1\end{center}
    \vskip0.5em}%

}
\usepackage{hyperref}
\hypersetup{
colorlinks=true,
linkcolor=blue,
filecolor=magenta,      
urlcolor=blue,
citecolor=blue,
}

\urlstyle{same}


\title{Lecture 1 Notes}
\date{June 15, 2021}
\author{Michael Brodskiy\\ \small Instructor: Prof. Yoon}

% Mathematical Operations:

% Sum: $$\sum_{n=a}^{b} f(x) $$
% Integral: $$\int_{lower}^{upper} f(x) dx$$
% Limit: $$\lim_{x\to\infty} f(x)$$

\begin{document}

    \maketitle

    \begin{enumerate}

      \item Economics — The study of how people manage scarce resources to achieve their goals

      \item Economic Groups:

        \begin{enumerate}

          \item Individuals, Households, and Consumers — Consume goods or services and provide factors of production

            \begin{enumerate}

              \item Goals: Maximize utility from consuming goods or services and providing factors of production

              \item Resources: Income to purchase goods or services, time

            \end{enumerate}

          \item Firms, Entrepreneurs, Businesses, Companies, and Producers — Produce goods or services and use factors of production
            
            \begin{enumerate}

              \item Goals: Maximize profit from producing and selling goods or services and using factors of production

              \item Resources: Labor, Natural Resources, Capital, and Entrepreneurship

            \end{enumerate}

          \item Governments — Spend their budget to implement their policies or social welfare programs

            \begin{enumerate}

              \item Goals: Maximize the well-being of consumers and producers

              \item Resources: Budget

            \end{enumerate}

        \end{enumerate}

      \item Three Economic Assumptions:

        \begin{enumerate}

          \item People are rational

            \begin{enumerate}

              \item People make decisions to achieve their highest goals given scarce resources and available information

              \item Economists assume that people are rational on average even though not everyone behaves rationally

              \item Ex. Individuals consume goods or services to maximize their utility given income and prices of goods or services

            \end{enumerate}

          \item People respond to incentives

            \begin{enumerate}

              \item An incentive is something that causes a change in the trade-offs that people face

              \item Incentive involves total benefit and total cost

                \begin{enumerate}

                  \item Total Benefit — Total amount of gain by doing the activity

                  \item Total Cost — Total amount paid or forfeited to do the activity

                \end{enumerate}

              \item Positive incentives makes people do more of an activity, while negative incentives (disincentives) decrease likelihood of performing an activity

            \end{enumerate}

          \item People make marginal decisions

            \begin{enumerate}

              \item Rational people make decisions by comparing marginal benefit to marginal cost

                \begin{enumerate}

                  \item Marginal Benefit is the benefit of performing an additional unit of an activity: $B_M=\frac{TB_1-TB_0}{Q_1-Q_0}$, where $TB_1$ is the new total benefit, $TB_0$ is the old total benefit, $Q_1$ is the new amount of activity, and $Q_0$ is the old amount of activity

                  \item Marginal Cost is the cost of performing an additional unit of an activity: $C_M=\frac{TC_1-TC_0}{Q_1-Q_0}$, where $TC_1$ is the new total cost, $TC_0$ is the old total cost, $Q_1$ is the new amount of activity, and $Q_0$ is the old amount of activity

                \end{enumerate}

              \item Decision criteria depends on how much of an activity people take to maximize net benefit ($TB-TC$)

                \begin{enumerate}

                  \item If $B_M>C_M$, more of an activity should be done to increase net benefit

                  \item If $B_M<C_M$, less of an activity should be done to increase net benefit

                  \item If $B_M=C_M$, the level of activity should be maintained to keep benefit in equilibrium

                \end{enumerate}

              \item This process is known as the \textbf{marginal decision-making process}

            \end{enumerate}

        \end{enumerate}

      \item \textbf{Scarcity}:

        \begin{enumerate}

          \item Peoples' wants are unlimited or infinite, but resources are limited, scarce, or finite

          \item Peoples' wants are constrained by scarce resources

          \item People need to make a choice among alternatives

          \item Ex. Individuals need to make a choice among different products given income

        \end{enumerate}

      \item \textbf{Trade-off}:

        \begin{enumerate}

          \item A trade-off occurs when people forfeit one activity to do another activity or when people give up more of one activity to get more of another activity

          \item Ex. The U.S. Government reduces its spending on education and social welfare to raise its defense spending

        \end{enumerate}

      \item \textbf{Opportunity Cost}:

        \begin{enumerate}

          \item The opportunity cost is the value of the second-best option given up to choose the best option

          \item It is the quantitative measure of trade-off relationships among alternative options

          \item Economists often express opportunity cost as a dollar value to compare alternatives

          \item Not all opportunity costs are measured in dollar or monetary values

          \item Ex. Toyota can produce 2 Corollas or 1 Camry with the same resources. What is the opportunity cost of 1 Camry? 2 Corollas

        \end{enumerate}

      \item \textbf{Efficiency}:

        \begin{enumerate}

          \item Efficiency occurs when resources are used to create the greatest economic value within a society

          \item Productive efficiency occurs when goods or services are produced at the minimum cost or when goods or services are produced to maximize profit

          \item Allocative efficiency occurs when goods or services produced satisfy consumers' preferences most or when consumers obtain goods or services to maximize their efficiency

        \end{enumerate}

      \item \textbf{Economic Growth}:

        \begin{enumerate}

          \item Economic growth is an increase in the production of goods or services

          \item Three ways to obtain economic growth exist:

            \begin{enumerate}

              \item Increase resources (labor, capital, natural resources, and entrepreneurship)

              \item Technological advancement (progress) or positive technological change or innovation

              \item Specialization and trade — Gains from trade with the right terms of trade

            \end{enumerate}

        \end{enumerate}

      \item \textbf{Economic Model}

        \begin{enumerate}

          \item A simplified representation or version of a real economic phenomenon (situation)

          \item Focuses on essentials of the complex reality by simplifying through assumptions, getting useful and approximate answers to economic problems, and obtaining predictions for the future

          \item May be expressed in three ways: words, graphs, and equations

          \item Ex. Circular flow model (words and diagrams) and Production possibilites frontier (in words, graphs, and equations)

        \end{enumerate}

      \item Characteristics of a good model

        \begin{enumerate}

          \item Makes clear and reasonable assumptions (based on economic theory and data)

          \item Predicts cause and effect relationship among economic variables (focuses on causal relationship between two economic variables, assuming all else to be constant)

          \item Accurately describes the real world (data)

            \begin{enumerate}

              \item Does not need to include all complex details from reality

              \item Needs to approximately replicate reality

            \end{enumerate}

        \end{enumerate}

      \item Developing a good economic model

        \begin{enumerate}

          \item Make reasonable assumptions based on theory

          \item Formulate a testable hypothesis\footnote{A statement about causal relationships between two economic variables that may be true or not}

          \item Test a hypothesis through data collection and statistical models: null hypothesis (the hypothesis one wants to accept) and alternative hypothesis (hypothesis one wants to reject)

          \item Revise the model by changing assumptions, data, or both if the null hypothesis is rejected

          \item Use the revised model to explain or predict economic events

        \end{enumerate}

      \item Developing a demand model

        \begin{enumerate}

          \item Assume that everything else is constant, other than the price of a specific product

          \item Theoretical model: $P=a+bQ_d$, where $P$ is price, and $Q_d$ is the quantity determined

          \item Null hypothesis ($H_0$): There is a negative relationship between price and quantity determined ($b<0$)

          \item Alternative hypothesis ($H_A$): There is either no relationship or a positive relationship between the price and quantity determined ($b\geq0$)

          \item Statistical model: $P=a+bQ_d+\varepsilon$, where $\varepsilon$ is the error term

          \item Collect data on price and quantity demanded to estimate $b$

          \item If data rejects the null hypothesis, revise the model by changing data or assumptions. Collect data once again on price and quantity demanded to estimate $b$

          \item Ex. $P=10-2Q_d$ — If the quantity demanded increases by 1, the price will decline by $\$ 2$, or, if the price rises by $\$ 1$, the quantity demanded will decline by $.5$

        \end{enumerate}

      \item Economic Model Examples

        \begin{enumerate}

          \item Basic Circular Flow Model

            \begin{enumerate}

              \item Assumptions

                \begin{enumerate}

                  \item Two economic agents: households and firms

                  \item Two markets: product market and factor market

                \end{enumerate}

              \item One of the most basic models that shows the flow of money, goods and services, and inputs

              \item In the product market, households spend money to buy goods and services that firms obtain revenue by making and selling

              \item In the factor market, households earn income (wage, rent, and profit) by providing inputs that firms pay for (cost) to produce goods and services

            \end{enumerate}

            \begin{center}
              \begin{figure}[h]
                \centering
                \tikzset{every picture/.style={line width=0.75pt}} %set default line width to 0.75pt        

\begin{tikzpicture}[x=0.75pt,y=0.75pt,yscale=-1,xscale=1]
%uncomment if require: \path (0,300); %set diagram left start at 0, and has height of 300

%Rounded Rect [id:dp3465100630073805] 
\draw  [fill={rgb, 255:red, 74; green, 144; blue, 226 }  ,fill opacity=1 ] (263,22.2) .. controls (263,16.57) and (267.57,12) .. (273.2,12) -- (389.8,12) .. controls (395.43,12) and (400,16.57) .. (400,22.2) -- (400,52.8) .. controls (400,58.43) and (395.43,63) .. (389.8,63) -- (273.2,63) .. controls (267.57,63) and (263,58.43) .. (263,52.8) -- cycle ;
%Rounded Rect [id:dp7214464129129555] 
\draw  [fill={rgb, 255:red, 255; green, 255; blue, 255 }  ,fill opacity=1 ] (273.5,25.5) .. controls (273.5,21.08) and (277.08,17.5) .. (281.5,17.5) -- (381.5,17.5) .. controls (385.92,17.5) and (389.5,21.08) .. (389.5,25.5) -- (389.5,49.5) .. controls (389.5,53.92) and (385.92,57.5) .. (381.5,57.5) -- (281.5,57.5) .. controls (277.08,57.5) and (273.5,53.92) .. (273.5,49.5) -- cycle ;
%Rounded Rect [id:dp8133849265641938] 
\draw  [fill={rgb, 255:red, 74; green, 144; blue, 226 }  ,fill opacity=1 ] (458,124.2) .. controls (458,118.57) and (462.57,114) .. (468.2,114) -- (584.8,114) .. controls (590.43,114) and (595,118.57) .. (595,124.2) -- (595,154.8) .. controls (595,160.43) and (590.43,165) .. (584.8,165) -- (468.2,165) .. controls (462.57,165) and (458,160.43) .. (458,154.8) -- cycle ;
%Rounded Rect [id:dp30443809253463616] 
\draw  [fill={rgb, 255:red, 255; green, 255; blue, 255 }  ,fill opacity=1 ] (468.5,127.5) .. controls (468.5,123.08) and (472.08,119.5) .. (476.5,119.5) -- (576.5,119.5) .. controls (580.92,119.5) and (584.5,123.08) .. (584.5,127.5) -- (584.5,151.5) .. controls (584.5,155.92) and (580.92,159.5) .. (576.5,159.5) -- (476.5,159.5) .. controls (472.08,159.5) and (468.5,155.92) .. (468.5,151.5) -- cycle ;
%Rounded Rect [id:dp5907574749667166] 
\draw  [fill={rgb, 255:red, 74; green, 144; blue, 226 }  ,fill opacity=1 ] (89,125.2) .. controls (89,119.57) and (93.57,115) .. (99.2,115) -- (215.8,115) .. controls (221.43,115) and (226,119.57) .. (226,125.2) -- (226,155.8) .. controls (226,161.43) and (221.43,166) .. (215.8,166) -- (99.2,166) .. controls (93.57,166) and (89,161.43) .. (89,155.8) -- cycle ;
%Rounded Rect [id:dp5310892985635383] 
\draw  [fill={rgb, 255:red, 255; green, 255; blue, 255 }  ,fill opacity=1 ] (99.5,128.5) .. controls (99.5,124.08) and (103.08,120.5) .. (107.5,120.5) -- (207.5,120.5) .. controls (211.92,120.5) and (215.5,124.08) .. (215.5,128.5) -- (215.5,152.5) .. controls (215.5,156.92) and (211.92,160.5) .. (207.5,160.5) -- (107.5,160.5) .. controls (103.08,160.5) and (99.5,156.92) .. (99.5,152.5) -- cycle ;
%Rounded Rect [id:dp827002824816521] 
\draw  [fill={rgb, 255:red, 74; green, 144; blue, 226 }  ,fill opacity=1 ] (263,237.2) .. controls (263,231.57) and (267.57,227) .. (273.2,227) -- (389.8,227) .. controls (395.43,227) and (400,231.57) .. (400,237.2) -- (400,267.8) .. controls (400,273.43) and (395.43,278) .. (389.8,278) -- (273.2,278) .. controls (267.57,278) and (263,273.43) .. (263,267.8) -- cycle ;
%Rounded Rect [id:dp4820855032011546] 
\draw  [fill={rgb, 255:red, 255; green, 255; blue, 255 }  ,fill opacity=1 ] (273.5,240.5) .. controls (273.5,236.08) and (277.08,232.5) .. (281.5,232.5) -- (381.5,232.5) .. controls (385.92,232.5) and (389.5,236.08) .. (389.5,240.5) -- (389.5,264.5) .. controls (389.5,268.92) and (385.92,272.5) .. (381.5,272.5) -- (281.5,272.5) .. controls (277.08,272.5) and (273.5,268.92) .. (273.5,264.5) -- cycle ;
%Straight Lines [id:da22766006944212214] 
\draw [color={rgb, 255:red, 208; green, 2; blue, 27 }  ,draw opacity=1 ][fill={rgb, 255:red, 208; green, 2; blue, 27 }  ,fill opacity=1 ][line width=2.25]    (259.98,56.78) -- (215.8,115) ;
\draw [shift={(263,52.8)}, rotate = 127.19] [fill={rgb, 255:red, 208; green, 2; blue, 27 }  ,fill opacity=1 ][line width=0.08]  [draw opacity=0] (16.07,-7.72) -- (0,0) -- (16.07,7.72) -- (10.67,0) -- cycle    ;
%Straight Lines [id:da10983822042305391] 
\draw [color={rgb, 255:red, 126; green, 211; blue, 33 }  ,draw opacity=1 ][fill={rgb, 255:red, 126; green, 211; blue, 33 }  ,fill opacity=1 ][line width=2.25]    (263,22.2) -- (103.55,112.54) ;
\draw [shift={(99.2,115)}, rotate = 330.47] [fill={rgb, 255:red, 126; green, 211; blue, 33 }  ,fill opacity=1 ][line width=0.08]  [draw opacity=0] (16.07,-7.72) -- (0,0) -- (16.07,7.72) -- (10.67,0) -- cycle    ;
%Straight Lines [id:da13493886382595943] 
\draw [color={rgb, 255:red, 126; green, 211; blue, 33 }  ,draw opacity=1 ][fill={rgb, 255:red, 126; green, 211; blue, 33 }  ,fill opacity=1 ][line width=2.25]    (99.2,166) -- (258.75,265.16) ;
\draw [shift={(263,267.8)}, rotate = 211.86] [fill={rgb, 255:red, 126; green, 211; blue, 33 }  ,fill opacity=1 ][line width=0.08]  [draw opacity=0] (16.07,-7.72) -- (0,0) -- (16.07,7.72) -- (10.67,0) -- cycle    ;
%Straight Lines [id:da7517724119578088] 
\draw [color={rgb, 255:red, 208; green, 2; blue, 27 }  ,draw opacity=1 ][fill={rgb, 255:red, 208; green, 2; blue, 27 }  ,fill opacity=1 ][line width=2.25]    (219.23,169.64) -- (273.2,227) ;
\draw [shift={(215.8,166)}, rotate = 46.74] [fill={rgb, 255:red, 208; green, 2; blue, 27 }  ,fill opacity=1 ][line width=0.08]  [draw opacity=0] (16.07,-7.72) -- (0,0) -- (16.07,7.72) -- (10.67,0) -- cycle    ;
%Straight Lines [id:da36357253787164334] 
\draw [color={rgb, 255:red, 208; green, 2; blue, 27 }  ,draw opacity=1 ][fill={rgb, 255:red, 208; green, 2; blue, 27 }  ,fill opacity=1 ][line width=2.25]    (393.72,223.9) -- (468.2,165) ;
\draw [shift={(389.8,227)}, rotate = 321.66] [fill={rgb, 255:red, 208; green, 2; blue, 27 }  ,fill opacity=1 ][line width=0.08]  [draw opacity=0] (16.07,-7.72) -- (0,0) -- (16.07,7.72) -- (10.67,0) -- cycle    ;
%Straight Lines [id:da17368072688608072] 
\draw [color={rgb, 255:red, 208; green, 2; blue, 27 }  ,draw opacity=1 ][fill={rgb, 255:red, 208; green, 2; blue, 27 }  ,fill opacity=1 ][line width=2.25]    (400,52.8) -- (464.48,110.66) ;
\draw [shift={(468.2,114)}, rotate = 221.9] [fill={rgb, 255:red, 208; green, 2; blue, 27 }  ,fill opacity=1 ][line width=0.08]  [draw opacity=0] (16.07,-7.72) -- (0,0) -- (16.07,7.72) -- (10.67,0) -- cycle    ;
%Straight Lines [id:da6621718953660283] 
\draw [color={rgb, 255:red, 126; green, 211; blue, 33 }  ,draw opacity=1 ][fill={rgb, 255:red, 126; green, 211; blue, 33 }  ,fill opacity=1 ][line width=2.25]    (400,267.8) -- (580.43,167.43) ;
\draw [shift={(584.8,165)}, rotate = 510.91] [fill={rgb, 255:red, 126; green, 211; blue, 33 }  ,fill opacity=1 ][line width=0.08]  [draw opacity=0] (16.07,-7.72) -- (0,0) -- (16.07,7.72) -- (10.67,0) -- cycle    ;
%Straight Lines [id:da4548385579774181] 
\draw [color={rgb, 255:red, 126; green, 211; blue, 33 }  ,draw opacity=1 ][fill={rgb, 255:red, 126; green, 211; blue, 33 }  ,fill opacity=1 ][line width=2.25]    (404.48,24.42) -- (584.8,114) ;
\draw [shift={(400,22.2)}, rotate = 26.42] [fill={rgb, 255:red, 126; green, 211; blue, 33 }  ,fill opacity=1 ][line width=0.08]  [draw opacity=0] (16.07,-7.72) -- (0,0) -- (16.07,7.72) -- (10.67,0) -- cycle    ;
%Straight Lines [id:da014368688940361807] 
\draw [draw opacity=0]   (157.5,140.5) -- (242,141) ;
%Straight Lines [id:da27176174044752055] 
\draw [draw opacity=0]   (526.5,139.5) -- (437,139.5) ;

% Text Node
\draw (331.5,37.5) node   [align=left] {Households};
% Text Node
\draw (157.5,140.5) node   [align=left] {Product Market};
% Text Node
\draw (526.5,139.5) node   [align=left] {Factor Market};
% Text Node
\draw (331.5,252.5) node   [align=left] {Firms};
% Text Node
\draw (244,141) node [anchor=west] [inner sep=0.75pt]   [align=left] {Goods or\\Services};
% Text Node
\draw (435,139.5) node [anchor=east] [inner sep=0.75pt]   [align=left] {Inputs};
% Text Node
\draw (494.4,219.4) node [anchor=north west][inner sep=0.75pt]   [align=left] {Cost};
% Text Node
\draw (494.4,65.1) node [anchor=south west] [inner sep=0.75pt]   [align=left] {\begin{minipage}[lt]{92.5pt}\setlength\topsep{0pt}
\begin{center}
Income\\(Wage, Rent, Profit)
\end{center}

\end{minipage}};
% Text Node
\draw (179.1,219.9) node [anchor=north east] [inner sep=0.75pt]   [align=left] {Revenue};
% Text Node
\draw (179.1,65.6) node [anchor=south east] [inner sep=0.75pt]   [align=left] {Spending};


\end{tikzpicture}


                \caption{Example Basic Circular Flow Model}
                \label{fig:1}
              \end{figure}
            \end{center}

            \begin{enumerate}

              \item Missing Components

                \begin{enumerate}

                  \item Missing economic agents: Governments

                  \item Missing markets: financial and international market

                  \item A modified circular flow model includes these components

                \end{enumerate}

            \end{enumerate}

        \end{enumerate}

    \end{enumerate}

\end{document}

