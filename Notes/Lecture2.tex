%%%%%%%%%%%%%%%%%%%%%%%%%%%%%%%%%%%%%%%%%%%%%%%%%%%%%%%%%%%%%%%%%%%%%%%%%%%%%%%%%%%%%%%%%%%%%%%%%%%%%%%%%%%%%%%%%%%%%%%%%%%%%%%%%%%%%%%%%%%%%%%%%%%%%%%%%%%%%%%%%%%%%%%%%%%%%%%%%%%%%%%%%%%%
% Written By Michael Brodskiy
% Class: Principles of Macroeconomics
% Professor: H. Yoon
%%%%%%%%%%%%%%%%%%%%%%%%%%%%%%%%%%%%%%%%%%%%%%%%%%%%%%%%%%%%%%%%%%%%%%%%%%%%%%%%%%%%%%%%%%%%%%%%%%%%%%%%%%%%%%%%%%%%%%%%%%%%%%%%%%%%%%%%%%%%%%%%%%%%%%%%%%%%%%%%%%%%%%%%%%%%%%%%%%%%%%%%%%%%

\documentclass[12pt]{article} 
\usepackage{alphalph}
\usepackage[utf8]{inputenc}
\usepackage[russian,english]{babel}
\usepackage{titling}
\usepackage{amsmath}
\usepackage{graphicx}
\usepackage{enumitem}
\usepackage{amssymb}
\usepackage[super]{nth}
\usepackage{everysel}
\usepackage{ragged2e}
\usepackage{geometry}
\usepackage{fancyhdr}
\usepackage{cancel}
\usepackage{siunitx}
\usepackage{xcolor}
\usepackage{physics}
\usepackage{tikz}
\usepackage{mathdots}
\usepackage{yhmath}
\usepackage{color}
\usepackage{array}
\usepackage{multirow}
\usepackage{gensymb}
\usepackage{tabularx}
\usepackage{extarrows}
\usepackage{booktabs}
\usetikzlibrary{fadings}
\usetikzlibrary{patterns}
\usetikzlibrary{shadows.blur}
\usetikzlibrary{shapes}


\geometry{top=1.0in,bottom=1.0in,left=1.0in,right=1.0in}
\newcommand{\subtitle}[1]{%
  \posttitle{%
    \par\end{center}
    \begin{center}\large#1\end{center}
    \vskip0.5em}%

}
\usepackage{hyperref}
\hypersetup{
colorlinks=true,
linkcolor=blue,
filecolor=magenta,      
urlcolor=blue,
citecolor=blue,
}

\urlstyle{same}


\title{Lecture 2 Notes}
\date{June 23, 2021}
\author{Michael Brodskiy\\ \small Instructor: Prof. Yoon}

% Mathematical Operations:

% Sum: $$\sum_{n=a}^{b} f(x) $$
% Integral: $$\int_{lower}^{upper} f(x) dx$$
% Limit: $$\lim_{x\to\infty} f(x)$$

\begin{document}

    \maketitle

    \begin{enumerate}

      \item Gross Domestic Product

        \begin{enumerate}

          \item There are three major macroeconomic variables that measure the health of a national economy: total production, price level, and unemployment

          \item The most commonly used measure of the total production or total income for the whole nation is a gross domestic product (GDP)

          \item GDP is the sum of the market values of all final goods and services produced within a country during a specific period of time

          \item There are four main parts of GDP:

            \begin{enumerate}

              \item Market Value

                \begin{enumerate}

                  \item In summing the quantities of goods and services, market value is found by taking the sum of market price times market quantity of all products

                  \item Mathematically, this is: $\sum_i P_iQ_i$, where $P_i$ and $Q_i$ are the price and quantity of a product $i$, respectively

                \end{enumerate}

              \item Final Goods and Services

                \begin{enumerate}

                  \item Final products cannot be used to produce other products and end up with consumers (Ex. Cars and hamburgers)

                  \item Intermediate products can be used to produce other products (Ex. bread and vegetables)

                  \item The values of intermediate products are not included in GDP (because those values are included in finished products)

                \end{enumerate}

              \item Produced Within a Country

                \begin{enumerate}

                  \item GDP includes products produced within a country even if they are produced by foreign firms and sold in foreign countries (Ex. Toyota Camry made in the US)

                  \item GDP excludes products produced outside a country even if they are produced by domestic firms and sold in the US (Ex. BMW made in Germany and sold in the US)

                  \item Gross National Product (GNP) is the sum of the market values of all final goods and services produced by domestic firms during a specific period of time

                  \item GNP includes products produced by domestic firms even if they are produced outside of a country (EX. Ford Explorer made in Korea or the US)

                  \item GNP excludes products produced by foreign firms even if they are produced within a country (Ex. Toyota Camry made in the US or Japan)

                \end{enumerate}

              \item Specific Period of Time

                \begin{enumerate}

                  \item GDP is normally calculated quarterly but is annualized by taking seasonal patterns into account from the quarterly GDP

                  \item This annualized GDP is called a seasonally adjusted annual GDP

                  \item GDP includes inventories, which are products that are produced but not sold in a given year

                  \item Increase in inventory increases GDP, and vice versa (Ex. Ford Mercury made in 2019 but sold in 2020 would be included in the 2019 GDP)

                \end{enumerate}

            \end{enumerate}

        \end{enumerate}

      \item Measuring GDP

        \begin{enumerate}

          \item The US Bureau of Economic Analysis (BEA) in the Department of Commerce measures GDP

          \item The BEA reports nominal GDP, real GDP, GDP deflator, growth rate, and other measures of income in the National Income and Product Accounts (NIPA) quarterly and annually

          \item GDP is measured in three approaches: expenditure, income, and value-added

          \item Finding GDP from a Modified Circular Flow Model:

            \begin{enumerate}

              \item A modified circular flow model is built by adding the government, financial market, and international market to a basic circular flow model

              \item The government earns tax revenue from firms and households and uses it to purchase goods or services from firms or provide transfer payments such as unemployment insurance and social security to households

              \item In the international market, foreign buyers purchase domestic goods or services, and domestic buyers purchase foreign goods or services

              \item A modified circular flow model shows that the expenditure, income, and the value-added are equal to one another

              \item The sum of the market values of all final goods and services is calculated by the expenditure on goods and services by households, firms, government, and foreign consumers

              \item Total income consists of wage for labor, rent for capital and natural resources, profit for entrepreneurs, interest for savers, and tax for government

              \item Value-added is the sum of the value of final products, minus the value of intermediate products

            \end{enumerate}

        \end{enumerate}

                \begin{figure}[h!]
                  \centering
                  \tikzset{every picture/.style={line width=0.75pt}} %set default line width to 0.75pt        

\begin{tikzpicture}[x=0.75pt,y=0.75pt,yscale=-1,xscale=1]
%uncomment if require: \path (0,675); %set diagram left start at 0, and has height of 675

%Straight Lines [id:da4915745116282617] 
\draw [color={rgb, 255:red, 208; green, 2; blue, 27 }  ,draw opacity=1 ][line width=2.25]    (262,440) -- (261.61,258.4) ;
\draw [shift={(261.6,254.4)}, rotate = 449.88] [color={rgb, 255:red, 208; green, 2; blue, 27 }  ,draw opacity=1 ][line width=2.25]    (17.49,-5.26) .. controls (11.12,-2.23) and (5.29,-0.48) .. (0,0) .. controls (5.29,0.48) and (11.12,2.23) .. (17.49,5.26)   ;
%Rounded Rect [id:dp7666180589810684] 
\draw  [color={rgb, 255:red, 126; green, 211; blue, 33 }  ,draw opacity=1 ][fill={rgb, 255:red, 208; green, 2; blue, 27 }  ,fill opacity=1 ][line width=2.25]  (250,22) .. controls (250,15.37) and (255.37,10) .. (262,10) -- (388,10) .. controls (394.63,10) and (400,15.37) .. (400,22) -- (400,58) .. controls (400,64.63) and (394.63,70) .. (388,70) -- (262,70) .. controls (255.37,70) and (250,64.63) .. (250,58) -- cycle ;
%Rounded Rect [id:dp4036980335074313] 
\draw  [fill={rgb, 255:red, 255; green, 255; blue, 255 }  ,fill opacity=1 ] (267.5,28) .. controls (267.5,23.58) and (271.08,20) .. (275.5,20) -- (374.5,20) .. controls (378.92,20) and (382.5,23.58) .. (382.5,28) -- (382.5,52) .. controls (382.5,56.42) and (378.92,60) .. (374.5,60) -- (275.5,60) .. controls (271.08,60) and (267.5,56.42) .. (267.5,52) -- cycle ;
%Rounded Rect [id:dp0076034584940721794] 
\draw  [color={rgb, 255:red, 126; green, 211; blue, 33 }  ,draw opacity=1 ][fill={rgb, 255:red, 208; green, 2; blue, 27 }  ,fill opacity=1 ][line width=2.25]  (250,102) .. controls (250,95.37) and (255.37,90) .. (262,90) -- (388,90) .. controls (394.63,90) and (400,95.37) .. (400,102) -- (400,138) .. controls (400,144.63) and (394.63,150) .. (388,150) -- (262,150) .. controls (255.37,150) and (250,144.63) .. (250,138) -- cycle ;
%Rounded Rect [id:dp056157943156243895] 
\draw  [fill={rgb, 255:red, 255; green, 255; blue, 255 }  ,fill opacity=1 ] (267.5,108) .. controls (267.5,103.58) and (271.08,100) .. (275.5,100) -- (374.5,100) .. controls (378.92,100) and (382.5,103.58) .. (382.5,108) -- (382.5,132) .. controls (382.5,136.42) and (378.92,140) .. (374.5,140) -- (275.5,140) .. controls (271.08,140) and (267.5,136.42) .. (267.5,132) -- cycle ;
%Rounded Rect [id:dp5151237653010659] 
\draw  [color={rgb, 255:red, 126; green, 211; blue, 33 }  ,draw opacity=1 ][fill={rgb, 255:red, 208; green, 2; blue, 27 }  ,fill opacity=1 ][line width=2.25]  (250,312) .. controls (250,305.37) and (255.37,300) .. (262,300) -- (388,300) .. controls (394.63,300) and (400,305.37) .. (400,312) -- (400,348) .. controls (400,354.63) and (394.63,360) .. (388,360) -- (262,360) .. controls (255.37,360) and (250,354.63) .. (250,348) -- cycle ;
%Rounded Rect [id:dp4919703263733157] 
\draw  [fill={rgb, 255:red, 255; green, 255; blue, 255 }  ,fill opacity=1 ] (267.5,318) .. controls (267.5,313.58) and (271.08,310) .. (275.5,310) -- (374.5,310) .. controls (378.92,310) and (382.5,313.58) .. (382.5,318) -- (382.5,342) .. controls (382.5,346.42) and (378.92,350) .. (374.5,350) -- (275.5,350) .. controls (271.08,350) and (267.5,346.42) .. (267.5,342) -- cycle ;
%Rounded Rect [id:dp1883969180557037] 
\draw  [color={rgb, 255:red, 126; green, 211; blue, 33 }  ,draw opacity=1 ][fill={rgb, 255:red, 208; green, 2; blue, 27 }  ,fill opacity=1 ][line width=2.25]  (250,452) .. controls (250,445.37) and (255.37,440) .. (262,440) -- (388,440) .. controls (394.63,440) and (400,445.37) .. (400,452) -- (400,488) .. controls (400,494.63) and (394.63,500) .. (388,500) -- (262,500) .. controls (255.37,500) and (250,494.63) .. (250,488) -- cycle ;
%Rounded Rect [id:dp36860992135457404] 
\draw  [fill={rgb, 255:red, 255; green, 255; blue, 255 }  ,fill opacity=1 ] (267.5,458) .. controls (267.5,453.58) and (271.08,450) .. (275.5,450) -- (374.5,450) .. controls (378.92,450) and (382.5,453.58) .. (382.5,458) -- (382.5,482) .. controls (382.5,486.42) and (378.92,490) .. (374.5,490) -- (275.5,490) .. controls (271.08,490) and (267.5,486.42) .. (267.5,482) -- cycle ;
%Shape: Bevel [id:dp5990087675702268] 
\draw  [fill={rgb, 255:red, 74; green, 144; blue, 226 }  ,fill opacity=1 ] (250,196) -- (400,196) -- (400,254) -- (250,254) -- cycle ; \draw   (261.6,207.6) -- (388.4,207.6) -- (388.4,242.4) -- (261.6,242.4) -- cycle ; \draw   (250,196) -- (261.6,207.6) ; \draw   (400,196) -- (388.4,207.6) ; \draw   (400,254) -- (388.4,242.4) ; \draw   (250,254) -- (261.6,242.4) ;
%Shape: Rectangle [id:dp0907690551427337] 
\draw  [fill={rgb, 255:red, 255; green, 255; blue, 255 }  ,fill opacity=1 ] (261.6,207.6) -- (388.4,207.6) -- (388.4,242.4) -- (261.6,242.4) -- cycle ;
%Shape: Bevel [id:dp7384143585576907] 
\draw  [fill={rgb, 255:red, 74; green, 144; blue, 226 }  ,fill opacity=1 ] (483,196) -- (633,196) -- (633,254) -- (483,254) -- cycle ; \draw   (494.6,207.6) -- (621.4,207.6) -- (621.4,242.4) -- (494.6,242.4) -- cycle ; \draw   (483,196) -- (494.6,207.6) ; \draw   (633,196) -- (621.4,207.6) ; \draw   (633,254) -- (621.4,242.4) ; \draw   (483,254) -- (494.6,242.4) ;
%Shape: Rectangle [id:dp05480624962698799] 
\draw  [fill={rgb, 255:red, 255; green, 255; blue, 255 }  ,fill opacity=1 ] (494.6,207.6) -- (621.4,207.6) -- (621.4,242.4) -- (494.6,242.4) -- cycle ;
%Shape: Bevel [id:dp6569344098548685] 
\draw  [fill={rgb, 255:red, 74; green, 144; blue, 226 }  ,fill opacity=1 ] (17,196) -- (167,196) -- (167,254) -- (17,254) -- cycle ; \draw   (28.6,207.6) -- (155.4,207.6) -- (155.4,242.4) -- (28.6,242.4) -- cycle ; \draw   (17,196) -- (28.6,207.6) ; \draw   (167,196) -- (155.4,207.6) ; \draw   (167,254) -- (155.4,242.4) ; \draw   (17,254) -- (28.6,242.4) ;
%Shape: Rectangle [id:dp2770860085796285] 
\draw  [fill={rgb, 255:red, 255; green, 255; blue, 255 }  ,fill opacity=1 ] (28.6,207.6) -- (155.4,207.6) -- (155.4,242.4) -- (28.6,242.4) -- cycle ;
%Straight Lines [id:da8794565164851245] 
\draw [color={rgb, 255:red, 80; green, 227; blue, 194 }  ,draw opacity=1 ][line width=2.25]    (400,22) -- (596.38,192.97) ;
\draw [shift={(599.4,195.6)}, rotate = 221.04] [color={rgb, 255:red, 80; green, 227; blue, 194 }  ,draw opacity=1 ][line width=2.25]    (17.49,-5.26) .. controls (11.12,-2.23) and (5.29,-0.48) .. (0,0) .. controls (5.29,0.48) and (11.12,2.23) .. (17.49,5.26)   ;
%Straight Lines [id:da9341790912824395] 
\draw [color={rgb, 255:red, 80; green, 227; blue, 194 }  ,draw opacity=1 ][line width=2.25]    (50,195) -- (246.97,24.62) ;
\draw [shift={(250,22)}, rotate = 499.14] [color={rgb, 255:red, 80; green, 227; blue, 194 }  ,draw opacity=1 ][line width=2.25]    (17.49,-5.26) .. controls (11.12,-2.23) and (5.29,-0.48) .. (0,0) .. controls (5.29,0.48) and (11.12,2.23) .. (17.49,5.26)   ;
%Straight Lines [id:da2168731157078827] 
\draw [color={rgb, 255:red, 208; green, 2; blue, 27 }  ,draw opacity=1 ][line width=2.25]    (533,195) -- (403.28,104.29) ;
\draw [shift={(400,102)}, rotate = 394.96000000000004] [color={rgb, 255:red, 208; green, 2; blue, 27 }  ,draw opacity=1 ][line width=2.25]    (17.49,-5.26) .. controls (11.12,-2.23) and (5.29,-0.48) .. (0,0) .. controls (5.29,0.48) and (11.12,2.23) .. (17.49,5.26)   ;
%Straight Lines [id:da08136232984568303] 
\draw [color={rgb, 255:red, 208; green, 2; blue, 27 }  ,draw opacity=1 ][line width=2.25]    (250,102) -- (118.27,194.7) ;
\draw [shift={(115,197)}, rotate = 324.87] [color={rgb, 255:red, 208; green, 2; blue, 27 }  ,draw opacity=1 ][line width=2.25]    (17.49,-5.26) .. controls (11.12,-2.23) and (5.29,-0.48) .. (0,0) .. controls (5.29,0.48) and (11.12,2.23) .. (17.49,5.26)   ;
%Straight Lines [id:da2905211151230983] 
\draw [color={rgb, 255:red, 208; green, 2; blue, 27 }  ,draw opacity=1 ][line width=2.25]    (400,488) -- (597.39,258.04) ;
\draw [shift={(600,255)}, rotate = 490.64] [color={rgb, 255:red, 208; green, 2; blue, 27 }  ,draw opacity=1 ][line width=2.25]    (17.49,-5.26) .. controls (11.12,-2.23) and (5.29,-0.48) .. (0,0) .. controls (5.29,0.48) and (11.12,2.23) .. (17.49,5.26)   ;
%Straight Lines [id:da3461229381776626] 
\draw [color={rgb, 255:red, 208; green, 2; blue, 27 }  ,draw opacity=1 ][line width=2.25]    (560,254) -- (402.51,448.89) ;
\draw [shift={(400,452)}, rotate = 308.94] [color={rgb, 255:red, 208; green, 2; blue, 27 }  ,draw opacity=1 ][line width=2.25]    (17.49,-5.26) .. controls (11.12,-2.23) and (5.29,-0.48) .. (0,0) .. controls (5.29,0.48) and (11.12,2.23) .. (17.49,5.26)   ;
%Straight Lines [id:da3061754592126098] 
\draw [color={rgb, 255:red, 208; green, 2; blue, 27 }  ,draw opacity=1 ][line width=2.25]    (387.84,254.4) -- (388,436) ;
\draw [shift={(388,440)}, rotate = 269.95] [color={rgb, 255:red, 208; green, 2; blue, 27 }  ,draw opacity=1 ][line width=2.25]    (17.49,-5.26) .. controls (11.12,-2.23) and (5.29,-0.48) .. (0,0) .. controls (5.29,0.48) and (11.12,2.23) .. (17.49,5.26)   ;
%Straight Lines [id:da3070869598652378] 
\draw [color={rgb, 255:red, 208; green, 2; blue, 27 }  ,draw opacity=1 ][line width=2.25]    (168,240) -- (246,240) ;
\draw [shift={(250,240)}, rotate = 180] [color={rgb, 255:red, 208; green, 2; blue, 27 }  ,draw opacity=1 ][line width=2.25]    (17.49,-5.26) .. controls (11.12,-2.23) and (5.29,-0.48) .. (0,0) .. controls (5.29,0.48) and (11.12,2.23) .. (17.49,5.26)   ;
%Straight Lines [id:da5809686913093672] 
\draw [color={rgb, 255:red, 80; green, 227; blue, 194 }  ,draw opacity=1 ][line width=2.25]    (250,210) -- (172,210) ;
\draw [shift={(168,210)}, rotate = 360] [color={rgb, 255:red, 80; green, 227; blue, 194 }  ,draw opacity=1 ][line width=2.25]    (17.49,-5.26) .. controls (11.12,-2.23) and (5.29,-0.48) .. (0,0) .. controls (5.29,0.48) and (11.12,2.23) .. (17.49,5.26)   ;
%Straight Lines [id:da391877482920292] 
\draw [color={rgb, 255:red, 208; green, 2; blue, 27 }  ,draw opacity=1 ][line width=2.25]    (482,240) -- (404,240) ;
\draw [shift={(400,240)}, rotate = 360] [color={rgb, 255:red, 208; green, 2; blue, 27 }  ,draw opacity=1 ][line width=2.25]    (17.49,-5.26) .. controls (11.12,-2.23) and (5.29,-0.48) .. (0,0) .. controls (5.29,0.48) and (11.12,2.23) .. (17.49,5.26)   ;
%Straight Lines [id:da1286337433528154] 
\draw [color={rgb, 255:red, 80; green, 227; blue, 194 }  ,draw opacity=1 ][line width=2.25]    (400,210) -- (478,210) ;
\draw [shift={(482,210)}, rotate = 180] [color={rgb, 255:red, 80; green, 227; blue, 194 }  ,draw opacity=1 ][line width=2.25]    (17.49,-5.26) .. controls (11.12,-2.23) and (5.29,-0.48) .. (0,0) .. controls (5.29,0.48) and (11.12,2.23) .. (17.49,5.26)   ;
%Straight Lines [id:da06755104247379251] 
\draw [color={rgb, 255:red, 80; green, 227; blue, 194 }  ,draw opacity=1 ][line width=2.25]    (167,196) -- (246.72,140.29) ;
\draw [shift={(250,138)}, rotate = 505.05] [color={rgb, 255:red, 80; green, 227; blue, 194 }  ,draw opacity=1 ][line width=2.25]    (17.49,-5.26) .. controls (11.12,-2.23) and (5.29,-0.48) .. (0,0) .. controls (5.29,0.48) and (11.12,2.23) .. (17.49,5.26)   ;
%Straight Lines [id:da8606748777418821] 
\draw [color={rgb, 255:red, 80; green, 227; blue, 194 }  ,draw opacity=1 ][line width=2.25]    (483,196) -- (403.28,140.29) ;
\draw [shift={(400,138)}, rotate = 394.95] [color={rgb, 255:red, 80; green, 227; blue, 194 }  ,draw opacity=1 ][line width=2.25]    (17.49,-5.26) .. controls (11.12,-2.23) and (5.29,-0.48) .. (0,0) .. controls (5.29,0.48) and (11.12,2.23) .. (17.49,5.26)   ;
%Straight Lines [id:da3019965869467873] 
\draw [color={rgb, 255:red, 208; green, 2; blue, 27 }  ,draw opacity=1 ][line width=2.25]    (50,255) -- (247.39,484.96) ;
\draw [shift={(250,488)}, rotate = 229.36] [color={rgb, 255:red, 208; green, 2; blue, 27 }  ,draw opacity=1 ][line width=2.25]    (17.49,-5.26) .. controls (11.12,-2.23) and (5.29,-0.48) .. (0,0) .. controls (5.29,0.48) and (11.12,2.23) .. (17.49,5.26)   ;
%Straight Lines [id:da024490127240861415] 
\draw [color={rgb, 255:red, 208; green, 2; blue, 27 }  ,draw opacity=1 ][line width=2.25]    (250,452) -- (92.51,257.11) ;
\draw [shift={(90,254)}, rotate = 411.06] [color={rgb, 255:red, 208; green, 2; blue, 27 }  ,draw opacity=1 ][line width=2.25]    (17.49,-5.26) .. controls (11.12,-2.23) and (5.29,-0.48) .. (0,0) .. controls (5.29,0.48) and (11.12,2.23) .. (17.49,5.26)   ;
%Straight Lines [id:da6863366136238424] 
\draw [color={rgb, 255:red, 80; green, 227; blue, 194 }  ,draw opacity=1 ][line width=2.25]    (483,254) -- (403.32,307.76) ;
\draw [shift={(400,310)}, rotate = 325.99] [color={rgb, 255:red, 80; green, 227; blue, 194 }  ,draw opacity=1 ][line width=2.25]    (17.49,-5.26) .. controls (11.12,-2.23) and (5.29,-0.48) .. (0,0) .. controls (5.29,0.48) and (11.12,2.23) .. (17.49,5.26)   ;
%Straight Lines [id:da4618789974010884] 
\draw [color={rgb, 255:red, 208; green, 2; blue, 27 }  ,draw opacity=1 ][line width=2.25]    (400,348) -- (515.86,256.48) ;
\draw [shift={(519,254)}, rotate = 501.69] [color={rgb, 255:red, 208; green, 2; blue, 27 }  ,draw opacity=1 ][line width=2.25]    (17.49,-5.26) .. controls (11.12,-2.23) and (5.29,-0.48) .. (0,0) .. controls (5.29,0.48) and (11.12,2.23) .. (17.49,5.26)   ;
%Straight Lines [id:da831971348374207] 
\draw [color={rgb, 255:red, 80; green, 227; blue, 194 }  ,draw opacity=1 ][line width=2.25]    (250,310) -- (170.32,256.24) ;
\draw [shift={(167,254)}, rotate = 394.01] [color={rgb, 255:red, 80; green, 227; blue, 194 }  ,draw opacity=1 ][line width=2.25]    (17.49,-5.26) .. controls (11.12,-2.23) and (5.29,-0.48) .. (0,0) .. controls (5.29,0.48) and (11.12,2.23) .. (17.49,5.26)   ;
%Straight Lines [id:da7377226940328963] 
\draw [color={rgb, 255:red, 208; green, 2; blue, 27 }  ,draw opacity=1 ][line width=2.25]    (131,254) -- (246.86,345.52) ;
\draw [shift={(250,348)}, rotate = 218.31] [color={rgb, 255:red, 208; green, 2; blue, 27 }  ,draw opacity=1 ][line width=2.25]    (17.49,-5.26) .. controls (11.12,-2.23) and (5.29,-0.48) .. (0,0) .. controls (5.29,0.48) and (11.12,2.23) .. (17.49,5.26)   ;
%Shape: Rectangle [id:dp36768797565120415] 
\draw   (17,520) -- (632,520) -- (632,660) -- (17,660) -- cycle ;
%Shape: Rectangle [id:dp1210689224160959] 
\draw  [color={rgb, 255:red, 0; green, 0; blue, 0 }  ,draw opacity=1 ][fill={rgb, 255:red, 245; green, 166; blue, 35 }  ,fill opacity=1 ] (17,520) -- (632,520) -- (632,550) -- (17,550) -- cycle ;

% Text Node
\draw (325,225) node   [align=left] {Government};
% Text Node
\draw (325,36) node   [align=left] {\begin{minipage}[lt]{61.73pt}\setlength\topsep{0pt}
\begin{center}
International \\Market
\end{center}

\end{minipage}};
% Text Node
\draw (325,120) node   [align=left] {\begin{minipage}[lt]{71.91pt}\setlength\topsep{0pt}
\begin{center}
Product Market
\end{center}

\end{minipage}};
% Text Node
\draw (325,330) node   [align=left] {Factor Market};
% Text Node
\draw (558,225) node   [align=left] {Households};
% Text Node
\draw (92,225) node   [align=left] {Firms};
% Text Node
\draw (324.5,535) node   [align=left] {Legend};
% Text Node
\draw (155.55,103.44) node [anchor=south] [inner sep=0.75pt]  [rotate=-319.27] [align=left] {Exports};
% Text Node
\draw (494.19,103.77) node [anchor=south] [inner sep=0.75pt]  [rotate=-40.38] [align=left] {Imports};
% Text Node
\draw (187.58,144.82) node [anchor=south] [inner sep=0.75pt]  [rotate=-323.31] [align=left] {Revenue};
% Text Node
\draw (461.55,143.91) node [anchor=south] [inner sep=0.75pt]  [rotate=-35.65] [align=left] {Spending};
% Text Node
\draw (210.5,180) node [anchor=north west][inner sep=0.75pt]   [align=left] {A};
% Text Node
\draw (422,179) node [anchor=north west][inner sep=0.75pt]   [align=left] {B};
% Text Node
\draw (210.5,279) node [anchor=south west] [inner sep=0.75pt]   [align=left] {C};
% Text Node
\draw (439.5,279) node [anchor=south east] [inner sep=0.75pt]   [align=left] {C};
% Text Node
\draw (195.84,305.89) node [anchor=north] [inner sep=0.75pt]  [rotate=-38.93] [align=left] {Cost};
% Text Node
\draw (453.99,306.03) node [anchor=north] [inner sep=0.75pt]  [rotate=-319.56] [align=left] {Income};
% Text Node
\draw (502,374.5) node [anchor=north west][inner sep=0.75pt]   [align=left] {D};
% Text Node
\draw (148,374.5) node [anchor=north east] [inner sep=0.75pt]   [align=left] {D};
% Text Node
\draw (362,389) node [anchor=north west][inner sep=0.75pt]   [align=left] {D};
% Text Node
\draw (172,350) node [anchor=south west] [inner sep=0.75pt]   [align=left] {E};
% Text Node
\draw (478,350) node [anchor=south east] [inner sep=0.75pt]   [align=left] {E};
% Text Node
\draw (271,391) node [anchor=north west][inner sep=0.75pt]   [align=left] {E};
% Text Node
\draw (19,553) node [anchor=north west][inner sep=0.75pt]   [align=left] {A. Goods, Services, and Government Purchases};
% Text Node
\draw (19,572) node [anchor=north west][inner sep=0.75pt]   [align=left] {B. Goods, Services, and Transfer Payments};
% Text Node
\draw (19,591) node [anchor=north west][inner sep=0.75pt]   [align=left] {C. Tax Inputs};
% Text Node
\draw (19,611) node [anchor=north west][inner sep=0.75pt]   [align=left] {D. Saving (Receiving Interest)};
% Text Node
\draw (19,631) node [anchor=north west][inner sep=0.75pt]   [align=left] {E. Borrowing (Paying Interest)};
% Text Node
\draw (325,466) node   [align=left] {\begin{minipage}[lt]{46.4pt}\setlength\topsep{0pt}
\begin{center}
Financial \\Market
\end{center}

\end{minipage}};


\end{tikzpicture}

                  \caption{An Example Modified Circular Flow Model}
                  \label{fig:1}
                \end{figure}

      \item Expenditure Approach

        \begin{enumerate}

          \item GDP can be measured by the sum of the expenditure on goods and services by households, firms, government, and foreign buyers

          \item Consumption is a household's expenditure on goods or services

            \newpage

          \item Durable goods last more than three years (Ex. Cars), while non-durable goods last less than three years (Ex. Food)

          \item Investment is a firm's expenditure on goods and services that consist of non-residential fixed investment, residential fixed investment, and change in inventories

            \begin{enumerate}

              \item Non-residential fixed investment is spending on buildings, machineries, tools, warehouses, and intellectual property

              \item Residential fixed investment is a household's spending on newly built houses, which is the only spending by households among investments

              \item Change in inventories is change in goods and services that are made in that year, but not sold the same year

                \begin{enumerate}

                  \item Increase in inventories occurs when new goods and services are produced but not sold the same year

                  \item Decrease in inventories occurs when old goods and services are sold in the same year

                  \item Planned investment is higher than actual investment with increase in inventories, and lower than actual investment with decrease in inventories, with the exception of investment in financial assets, such as stocks, bonds, and mutual funds, as they do not count as part of investments (because firms are not directly buying goods or services in return)

                \end{enumerate}

            \end{enumerate}

          \item Government purchases are a government's expenditure

            \begin{enumerate}

              \item State and local government purchases include spending on education, health and hospitals, police and corrections, and highways and roads

              \item Federal government purchases include defense, transportation, education, housing, law enforcement, and interest on payments of government debt

              \item Exception: GDP does not include transfer payments (social security, unemployment insurance, and other welfare payments) because the government does not receive goods in return

            \end{enumerate}

          \item Net export is the export minus import (or foreign buyers' expenditure on domestic goods and services minus domestic buyers' expenditure on foreign goods and services)

            \begin{enumerate}

              \item Export must be included in GDP because it includes goods and services produced within a country

              \item Import must be excluded from GDP because it includes goods and services produced outside of a country

              \item Trade surplus occurs when there is a positive net export, or when export is greater than import. On the other hand, trade deficit occurs when there is a negative net export, or when import is greater than export. Trade balance occurs when the export equals the input

            \end{enumerate}

          \item The GDP Equation

            \begin{enumerate}

              \item GDP ($Y$) = Consumption ($C$) + Investment ($I$) + Government purchases ($G$) + Net Export ($NX$)

              \item This shows that the value of total production is equal to total expenditure by households, firms, governments, and foreign buyers

            \end{enumerate}

        \end{enumerate}

      \item Income Approach

        \begin{enumerate}

          \item GDP can also be measured by the sum of income earned by workers, land and property owners, managers, savers (financial asset owners), and government

          \item Total income, which is called Gross Domestic Income (GDI), consists of wages for workers, rents for land and property owners, profits for firm owners, interests for financial asset owners, taxes for the government, and statistical discrepancy ($GDP-GDI$)

          \item $GDI$ = Total income = Wage + Rent + Profit + Interest + Tax + Statistical Discrepancy

          \item Depreciation (consumption of capital) is the decline in value of capital in the NIPA account

          \item Statistical discrepancy refers to the difference between GDP and GDI due to measurement errors

        \end{enumerate}

      \item Value-Added Approach

        \begin{enumerate}

          \item GDP can also be measured by the sum of value-added by all firms that produce final goods and services

          \item The value-added is the market value firms add to goods and services and is calculated by subtracting the prices of intermediate goods and services from prices of final goods and services

          \item In the calculation of GDP, we use the sum of value-added to avoid double counting (just like we use the sum of the market values of all final goods and services, not including intermediate goods and services)

        \end{enumerate}

      \item Different GDP Measures

        \begin{enumerate}

          \item Nominal GDP

            \begin{enumerate}

              \item The sum of the market values of all final goods and services evaluated at the current year prices

              \item Mathematically calculated as $\sum_i P_{it}Q_{it}$, where $P_{it}$ is the price of a product $i$ at year $t$ and $Q_{it}$ is the quantity of a product $i$ at year $t$

              \item Nominal GDP reflects both the change in price level and change in quantity produced over time

              \item The problem of nominal GDP is that it does not take into account the change in price level over time

            \end{enumerate}

          \item Real GDP

            \begin{enumerate}

              \item The sum of the market values of all final goods and services evaluated at the base year prices

              \item Mathematically calculated as $\sum_i P_{ib}Q_{it}$, where $P_{ib}$ is the price of a product $i$ at the base year and $Q_{it}$ is the quantity produced at year $t$

              \item Real GDP reflects only the change in quantity over time but controls the change in price level

              \item Real GDP adjusts Nominal GDP for the price change

              \item Inflation is an increase in price level over time, while deflation is a decrease in price level over time

            \end{enumerate}

          \item \begin{tabular}{| c | c | c |} \hline & Inflation & Deflation\\ \hline Before base year & RGDP $>$ NGDP & RGDP $<$ NGDP\\ \hline Base year & RGDP = NGDP & RGDP = NGDP\\ \hline After base year & RGDP $<$ NGDP & RGDP $>$ NGDP\\\hline  \end{tabular}

        \end{enumerate}

      \item Limitations of GDP

        \begin{enumerate}

          \item Not a perfect measure of well-being

            \begin{enumerate}

              \item GDP does not perfectly measure the well-being (quality of life) of the society

              \item Economists develop a life satisfaction index that incorporates literacy rates, life expectancy, and child mortality to better measure the quality of life

            \end{enumerate}

          \item Exclusion of some economic activities

            \begin{enumerate}

              \item Goods and services that are not traded in a market include leisure and home production

              \item Goods and services that are not reported to the government include those in the black market and grey market

                \begin{enumerate}

                  \item Black Market — Goods and services illegally produced to avoid taxes and/or government regulations (Ex. Restricted weapons, human organs, etc.)

                  \item Grey Market — Goods and services that are legally produced, but not reported to the government (Ex. Mowing someone's lawn, Babysitting, etc.)

                \end{enumerate}

              \item Leisure, which is a non-labor activity, is supposed to be included in GDP because it increases GDP by improving productivity, but lowers GDP by reducing work hours

              \item Home production, or goods and services that are produced and consumed within a household, are not included in GDP

            \end{enumerate}

          \item Negative externalities

            \begin{enumerate}

              \item Negative externality occurs when a third party, which is not involved in market transactions (so neither a buyer nor seller), is negatively affected by economic activities

              \item Negative externality such as pollution generates external costs such as bad health effects

              \item It should be included in GDP, because it lowers it

              \item To take into account external cost, Green GDP is calculated as $GDP - $ external cost

            \end{enumerate}

          \item No consideration of equity

            \begin{enumerate}

              \item GDP considers the size of goods and services (efficiency), but does not care about the allocation of the goods and services (equity)

              \item Economic growth, which is an increase in the production of goods and services, may generate income inequality or inequity in a country

            \end{enumerate}

          \item Crime and other social problems

            \begin{enumerate}

              \item Higher GDP may cause higher crime rates and more social problems, such as high divorce rate, drug addiction, or racial issues, but these are not taken into account in GDP

            \end{enumerate}

        \end{enumerate}

      \item Economic Growth vs. Business Cycle

        \begin{enumerate}

          \item Economic growth is a long run increase in the production of goods and services in the economy or a long run increase in the average standard of living over time

          \item Business cycle is a short run fluctuation of the production of goods and services in the economy or a short run fluctuation of the average standard of living, as announced by the National Bureau of Economic Research (NBER)

          \item Business cycles range from 6 to 10 years, while economic growth usually occurs after greater than 10 years (usually 50-100 years)

        \end{enumerate}

      \item Measure of Average Standard of Living

        \begin{enumerate}

          \item Nominal GDP shows the change in price level and the change in quantity produced

          \item Real GDP adjusts nominal GDP for change in price level by fixing the price level at a base year

          \item Real GDP per capita adjusts real GDP for the difference in population among countries

          \item RGDP Per Capita: $\frac{RGDP}{Population}$

          \item RGDP per capita is used to compare economic growth across countries and over time because it adjusts for change in price level over time and difference in population across countries

        \end{enumerate}

      \item Measures of Economic Growth over time

        \begin{enumerate}

          \item There are two measures of economic growth over time: growth rate and time to double income

          \item Growth Rate (g)

            \begin{enumerate}

              \item An annual growth rate is the percentage change in real GDP per capita from year to year

              \item Measured as $g=\frac{Y_t-Y_{t-1}}{Y_{t-1}}\cdot100$, where $g$ is the annual growth rate, $Y_t$ is a real GDP per capita in year $t$, and $Y_{t-1}$ is a real GDP per capita in year $t-1$

              \item An average annual growth rate is the average annual percentage change in real GDP per capita from year 0 to year $t$

              \item Two ways to calculate average annual growth rate: accurate and approximate

                \begin{enumerate}

                  \item Annual average growth rate is measured using an accurate method as $g=100\left( e^{\frac{1}{t}\ln\left( \frac{Y_t}{Y_0} \right)}-1 \right)=100\left( \sqrt[t]{\frac{Y_t}{Y_0}}-1 \right)$, where $Y_t$ is the real GDP per capita in a final year, $Y_0$ is a real GDP per capita in an initial year, and $g$ is an average annual growth rate

                  \item The above formula is derived from $Y_t=Y_0(1+g)^t$

                  \item The approximate annual average growth rate can be found using $\frac{1}{t}\sum_{n=1}^t g_t$, where $g_t$ is an annual growth rate in percent

                  \item The approximate method can only be used when you know the annual growth rate for each year

                \end{enumerate}

            \end{enumerate}

          \item Time to double income

            \begin{enumerate}

              \item Time to double income can be measured in two ways: accurate and approximate

                \begin{enumerate}

                  \item Time to double income ($T$) can be measured by using an accurate method: $\frac{\ln(2)}{\ln(1+g)}$, where $g$ is an average annual growth rate (as a decimal)

                  \item The above formula is derived from $2Y_0=Y_0(1+g)^t$

                  \item The time to double income can be approximated by using $T=\frac{70}{g}$, where $g$ is an average annual growth rate, in percent. This is called the rule of seventy

                \end{enumerate}

            \end{enumerate}

        \end{enumerate}

      \item Economic Growth Models

        \begin{enumerate}

          \item Economic growth models are developed to explain the sources of economic growth and policy implications

          \item There are two major categories of economic growth models: Exogenous and Endogenous

          \item Solow Model (Exogenous Growth Model)

            \begin{enumerate}

              \item Developed by Robert Solow in 1950, and modified later

              \item Major sources of economic growth in the Solow model are accumulation of physical capital and technological advancement

              \item Uses the aggregate production function: $y = f$(physical capital, human capital, natural resources), where $y$ is a real GDP per capita (labor productivity or average income) and f() is a technology

              \item Labor productivity (real GDP per capita) is determined by physical capital, human capital, natural resources, and technology

              \item Physical capital is itself a product but is also used as an input to produce other products (Ex. Tools, Equipment, etc.)

              \item Investment in physical capital comes from saving

              \item Human capital is the knowledge, skill, and ability of workers and managers, through education, training, and experience, which is different from labor in that labor is measured by hours of work

              \item Knowledge generates a positive externality, which occurs when the market transactions positively affect a third party

              \item Natural resources include renewables that can be reproduced, such as trees, electricity, and water, and non-renewables that can not be reproduced, such as coal, oil, gold, and natural gas

              \item Technology is something that changes inputs to outputs and generates a positive externality

            \end{enumerate}

          \item Aggregate Production Function

            \begin{enumerate}

              \item Labor productivity increases at a decreasing rate, as the economy accumulates more physical capital

              \item The marginal product of physical capital is the increase in production as a result of increase in physical capital by one more unit, which is the slope of the aggregate production function at each point

              \item The marginal product of physical capital declines as the economy accumulates more physical capital

              \item This is called diminishing marginal product (returns) to capital

              \item Two implications of diminishing returns:

                \begin{enumerate}
                    
                  \item Steady-state is a situation where the economy stops accumualting capital or where marginal return to capital = marginal cost to capital (depreciation)

                  \item Catch-up effect (convergence)

                    \begin{itemize}

                      \item Poor countries have a higher return to capital than richer countries because they have a lower physical capital and real GDP per capita

                      \item Poor countries try to accumulate more capital and have higher growth rates than rich countries

                      \item Real GDP per capita across countries will eventually be equalized because poor countries will catch up with rich countries

                    \end{itemize}

                \end{enumerate}

              \item Strong catch-up effect (convergence) in high income countries: strong negative relationship between initial real GDP per capita and average annual growth rate

              \item Weak catch-up effect (convergence) in all countries whose data is available: weak negative relationship with outliers such as Colombia, Niger, and Congo

            \end{enumerate}

          \item Aggregate Production Function with Technological Advance

            \begin{enumerate}

              \item $y=Af$(physical capital, human capital, natural resources), where $A$ is the technological change

              \item Without the technological change ($y_0$), the economy stops growing at the steady state $(B)$ due to diminishing returns

              \item Technological advance will shift the aggregate production curve from $y_0$ to $y_1$ (curve looks like a steeper root curve)

              \item The economy will delay the steady state from $B$ to $D$

              \item This will lead to sustainable economic growth

              \item In the Solow Model, technological advancement occurs exogenously outside the model through random scientific discoveries (thus it is called an exogenous model)

              \item Policy implications of the Solow Model

                \begin{enumerate}

                  \item Poor countries need to accumulate physical capital and advance their technologies through foreign aids such as direct or portfolio investments

                    \begin{itemize}

                      \item Foreign Direct Investment — Rich countries directly build or purchase facilities in poor countries

                      \item Foreign Portfolio Investment — Rich countries buy financial assets in poor countries

                    \end{itemize}

                \end{enumerate}

              \item Limitations of the Solow Model

                \begin{enumerate}

                  \item There are still poor countries, even with foreign aides (Ex. Haiti, Zimbabwe, etc.)

                  \item There are still countries with high growth rates, even without foreign aid (Ex. China, India, etc.)

                  \item To explain sources of economic growth, other than physical capital accumulation and technological advancement in these countries, new economic growth model called endogenous growth models were developed

                \end{enumerate}

            \end{enumerate}

          \item Endogenous Growth Model (developed by Paul Romer)

            \begin{enumerate}

              \item In this model, technological advancement occurs endogenously inside the model through efficient institutions

              \item Efficient institutions facilitate capital accumulation, technological advancement, and economic growth

              \item Aggregate production function with institutions: $y=Af$(physical capital, human capital, natural resources, institutions)

              \item Without efficient institutions, the economy stops growing at $B$ (steady state) with the aggregate production function of $y_0$

              \item With efficient institutions, the aggregate production function will shift up to $y_1$, with a higher Y-intercept, and the economy stops growing at $D$ (steady state)

              \item Efficient institutions not only help the economy to have higher capital accumulation, but also to develop better technology and have higher economic growth, even without foreign aid

              \item This explains any countries with high growth rates, even without foreign aid, and any countries with low growth rates, even with foreign aid

              \item There are seven major efficient institutions that facilitate capital accumulation, technological advancement, and economic growth:

                \begin{enumerate}

                  \item Private property rights

                    \begin{itemize}

                      \item Provides consumers with an incentive to pay for products and provide producers with an incentive to develop new technology and lower the cost (Ex. US has better property rights than China)

                    \end{itemize}

                  \item Political stability and rule of law

                    \begin{itemize}

                      \item Firms have more incentive to invest money in countries with polticial stability and countries that follow laws, rather than invest in poltically unstable countries (Ex. US and European countries are more stable than countries in Africa and Latin America)

                    \end{itemize}

                  \item Competitive market

                    \begin{itemize}

                      \item Has many buyers and sellers, sellers have no market power, there are no barriers to entry and exit, facilitates competition, and lowers cost of production, which results in maximum efficiency and economic growth

                    \end{itemize}

                  \item Free trade

                    \begin{itemize}

                      \item Specialization and free trade make countries have more products available for consumption with transfer of technology

                    \end{itemize}

                  \item Free flow of funds

                    \begin{itemize}

                      \item A foreign portfolio investment helps poor countries to develop only without barriers to flow of funds

                    \end{itemize}

                  \item Efficient taxes
                    
                    \begin{itemize}

                      \item Taxes normally generate tax revenue to the government and deadweight loss to the society

                      \item Efficient taxes are taxes that maximize tax revenue, but minimize deadweight loss

                    \end{itemize}

                  \item Stable price

                    \begin{itemize}

                      \item Firms have stronger incentive to invest their money in countries with stable price level due to predictable profits, rather than countries with significant change in price level

                    \end{itemize}

                \end{enumerate}

            \end{enumerate}

          \item Policy Implications of the Endogenous Growth Model

            \begin{enumerate}

              \item More efficient institutions are supported

            \end{enumerate}

        \end{enumerate}

      \item Price Level

        \begin{enumerate}

          \item Both income and price level normally rise over time

          \item If income and price level rise at the same rate, there is no change in purchasing power (ability to buy products) of consumers, because the real value of income does not change

          \item If price level rises at a faster rate than income, the purchasing power of consumers will decline because the real value of income declines

          \item If price level rises at a slower rate than income, the purchasing power of consumers will increase, because the real value of income increases

          \item Price level may be different across places even with the same income

        \end{enumerate}

      \item Different measures of price level

        \begin{enumerate}

          \item The cost of living for a country is measured by the price level, which is a weighted average price of goods and services produced within a country

          \item More weight is given to goods and services with a higher portion of expenditure, while less weight is given to goods and services with lower portion of expenditure

          \item Three major measures of price level:

            \begin{enumerate}

              \item GDP deflator

                \begin{enumerate}

                  \item GDP deflator = $\frac{NGDP}{RGDP}\cdot100$

                  \item It reflects only the change in price level because nominal GDP reflects both change in quantity and change in price level, while real GDP reflects only the change in quantity

                  \item This is the broadest measure of price level for a country because it reflects a weighted price for goods and services spent by households, firms, government, and foreigners

                  \item It is 100 at the base year, because $NGDP = RGDP$ at the base year

                  \item \begin{tabular}{|c|c|c|} \hline & Inflation & Deflation\\ \hline Before base year & GDP Deflator $<$ 100 & GDP Deflator $>$ 100\\ \hline Base year & GDP Deflator = 100 & GDP Deflator = 100\\ \hline After base year & GDP Deflator $>$ 100 & GDP Deflator $<$ 100\\\hline  \end{tabular}

                \end{enumerate}

              \item Producer price index (PPI)

                \begin{enumerate}

                  \item A weighted average price of all the final goods and services produced in the economy and spent by firms, such as machinery, equipment, tools, intellectual property, adn so on

                  \item Excludes the prices of goods and services spent by government and households

                  \item A predictor of consumer price index

                  \item CPI follows pattern changes in PPI

                \end{enumerate}

              \item Consumer price index (CPI)

                \begin{enumerate}

                  \item A weighted average of a market basket of consumer goods and services spent by typical urban consumers

                  \item The most commonly used price level for tracking changes in the cost of living in the US because it reflects the prices experienced by consumers

                  \item Typical urban households include all urban consumers, urban wage earners, and clerical workers (93\% of the US population)

                  \item Ex. Professionals, Self-employed, etc.

                  \item Exception: People living in rural areas, people in the armed forces, etc.

                  \item The market basket is a group of expenditure items spent by typical urban households in a base year (current base year is 1982-84)

                  \item It is fixed at the base year to capture only the change in price level over time

                  \item Classified into more than 200 categories and arranged into 8 major groups, which include housing, transportation, food and beverages, medical care, education and communication, recreation, apparel, and other goods and services

                  \item Government subsidies and sales and excise taxes are included in the market basket

                  \item The market basket excludes investment items (stocks, bonds, real estate, and life insurance), houses, antiques, collectibles, gambling losses, fines, cash gifts, child support, alimony, interest costs, illegal products, transfer payments, and income tax

                  \item CPI is measured in $CPI=\frac{Expenditure_t}{Expenditure_b}\cdot100$

                  \item $Expenditure_t$ is the expenditure in a current year ($\sum_i P_{it}Q_{ib}$, where $P_{it}$ is the price of a product $i$ at year $t$, and $Q_{ib}$ is the quantity of product $i$ at a base year)

                    \item $Expenditure_b$ is the expenditure in a base year ($\sum_i P_{ib}Q_{ib}$, where $P_{ib}$ is the price of a product $i$ in a base year)

                    \item Market basket is fixed at a base year, but prices change over time

                    \item At the base year, $CPI=100$ because $Expenditure_t=Expenditure_b$

                  \item \begin{tabular}{|c|c|c|} \hline & Inflation & Deflation\\ \hline Before base year & CPI $<$ 100 & CPI $>$ 100\\ \hline Base year & CPI = 100 & CPI = 100\\ \hline After base year & CPI $>$ 100 & CPI $<$ 100\\\hline  \end{tabular}

                \end{enumerate}

            \end{enumerate}

        \end{enumerate}

      \item Change in Price Level

        \begin{enumerate}

          \item Inflation rate is a year-to-year percentage change in price level over time

          \item Calculated as $\frac{P_t-P_{t-1}}{P_{t-1}}\cdot100$, where $P_t$ is the price level at year $t$ and $P_{t-1}$ is the price level at year $t-1$

          \item Inflation occurs if the rate is positive, and deflation occurs if the rate is negative

          \item There are two different inflation measures by using CPI: headline and core

            \begin{enumerate}

              \item Headline — A change in CPI for the entire market basket of the average urban consumer

              \item Core — A change in CPI for the market basket, excluding food and energy

            \end{enumerate}

          \item Including them in the calculation of CPI may over or under estimate the real change in overall prices

          \item Excluding them may miss a large part of income of consumers

        \end{enumerate}

      \item Uses of CPI

        \begin{enumerate}

          \item An economic indicator of inflation rate

            \begin{enumerate}

              \item CPI is the most widely used measure of inflation

              \item Provides information about change in price level in the national economy to government, business, labor, and private citizens, and is used as a guide to make economic decisions

              \item The President, Congress, and the Federal Reserve Board use trends in the CPI to aid in formulating fiscal and monetary policies

            \end{enumerate}

          \item Deflating nominal variables

            \begin{enumerate}

              \item CPI is used to adjust nominal variables such as nominal income for change in price level over time and convert them into real values

              \item Two dollar values in different years can be compared by converting the current value into future value, or vice versa

              \item Future Value in year $Y$ = Value in year $X$ $\cdot\frac{CPI_{year_Y}}{CPI_{year_X}}$

              \item Current Value in year $X$ = Value in year $Y$ $\cdot\frac{CPI_{year_X}}{CPI_{year_Y}}$

            \end{enumerate}

          \item Nominal interest rate to real interest rate

            \begin{enumerate}

              \item Nominal (stated) interest rate ($R$) is the cost of borrowing expressed as a percentage of the amount borrowed without adjusting for change in price level

              \item Real interest rate ($r$) is the true cost of borrowing that adjusts the nominal interest rate for the change in price level

              \item Two ways to convert between nominal and real interest rates:

                \begin{enumerate}

                  \item Accurate Method: $r=\frac{1+R}{1+\pi}-1$, where $\pi$ is an inflation rate. This method outputs a decimal value (multiply by 100 to find the percent)

                  \item Approximate Method: $r=R-\pi$. This outputs a percent value.

                    \begin{itemize}

                      \item $r>0$ if $R>\pi$ or $\pi<0$ (deflation)

                      \item $r<0$ if $R<\pi$

                    \end{itemize}

                \end{enumerate}

            \end{enumerate}

          \item Cost of Living Adjustment (COLA)

            \begin{enumerate}

              \item Indexing automatically increases payments in proportion to the cost of living

              \item Indexed payments are usually referred to as cost-of-living adjustments (COLA)

              \item Ex. Salaries and income with union contracts, social security payments, etc.

            \end{enumerate}

        \end{enumerate}

      \item Limitations of CPI

        \begin{enumerate}

          \item Substitution Bias

            \begin{enumerate}

              \item Consumers tend to substitute lower priced products for higher priced products over time

              \item CPI, however, fixes the market basket for 10 years

              \item CPI assumes that consumers keep buying the same amount of higher priced products, but keep the same amount of lower priced products for 10 years

              \item Thus, CPI is an overestimate of a true cost of living

            \end{enumerate}

          \item New Product Bias

            \begin{enumerate}

              \item Due to innovation, new products are developed over time

              \item CPI is supposed to include new products over time, but does not reflect the prices of new products because it updates the market basket every 10 years

            \end{enumerate}

          \item Outlet Bias

            \begin{enumerate}

              \item CPI could be an overestimate of true cost of living because it uses the retail prices of products from retail stores, not discounted prices from wholesale stores like Costco

            \end{enumerate}

          \item Quality Bias

            \begin{enumerate}

              \item The change in prices of products may reflect changes in their qualities, not the pure change of their prices

              \item The higher price of products may reflect their higher qualities, whereas the lower price of products may reflect their lower qualities

              \item As a result, CPI would be an overestimate

              \item BLS corrects this by attempting “hedonic quality adjustment” to consider quality changes

            \end{enumerate}

        \end{enumerate}

      \item Costs of Inflation

        \begin{enumerate}

          \item Costs of Anticipated inflation

            \begin{enumerate}

              \item Income redistribution — People with fixed future incomes will lose money with inflation because the value of their money will decline over time (thus, wages need to be adjusted for inflation by using COLA)

              \item Shoe-leather cost — People hold less money with inflation due to the loss of value over time. People tend to go to banks more often with inflation. It costs more to make trips to banks. This cost is called the shoe-leather cost since shoe and leather would wear out more easily with frequent travel to banks.

              \item Menu cost — With change in price level, firms need to reprint price catalogs more foten. The cost of changing price catalogs of menus is called the menu cost.

              \item Tax distortion — Capital gain tax (property tax) is based on the nominal value of financial assets (real estate). Capital gain tax is supposed to be lower with inflation due to decline in real value of financial assets, but rise in inflation

            \end{enumerate}

          \item Costs of Unanticipated inflation

            \begin{enumerate}

              \item Long term wage and mortgage loan contracts are based on the expected inflation rate

                \begin{enumerate}

                  \item If actual inflation $<$ expected inflation — Firms will lose because they pay more wages than they are supposed to, meaning borrowers will gain because they pay less interest rate than they are supposed to. Workers will gain because they are paid more wages than they are supposed to, and lenders lose because they receive less interest.

                  \item If actual inflation $>$ expected inflation — The opposite will happen

                \end{enumerate}

            \end{enumerate}

        \end{enumerate}

      \item Price Differences across Places

        \begin{enumerate}

          \item Purchasing power parity (PPP)

            \begin{enumerate}

              \item Occurs when any product costs the same everywhere, once the product's values are converted into a common currency (dollar) using foreign exchange rates under the following conditions:
                \begin{enumerate}

                  \item No transaction costs

                  \item No trade restrictions

                  \item All products are tradable

                \end{enumerate}

              \item PPP does not hold across places because:

                \begin{enumerate}

                  \item There are transaction costs, such as transportation costs and costs of finding sellers

                  \item There are trade restrictions, such as tariffs and import quotas

                  \item There are non-tradable products, such as apartments or haircuts

                \end{enumerate}

              \item Thus, purchasing power indexes are developed to adjust price differences across places

            \end{enumerate}

          \item Purchasing power indexes

            \begin{enumerate}

              \item Purchasing power indexes are developed to describe differences in prices across locations

              \item Steps to develop a purchasing power index:

                \begin{enumerate}

                  \item First, find a market basket of goods and services that can be compared across countries

                  \item Second, measure the prices of goods and services in the basket in each country, and calculate the overall cost of purchasing it in each country

                  \item Third, build an index showing how much the basket costs ineach country, relative to some base country

                \end{enumerate}

              \item The Big Mac Index is commonly used, because of the ubiquity of McDonald's

                \begin{enumerate}

                  \item Big Mac Index = $\frac{ER_{PPP}-ER_{Official}}{ER_{Official}}$, where $ER_{PPP}$ is the exchange rate between the price of a Big Mac in a foreign country and the price of a Big Mac in the US, assuming PPP holds, and $ER_{Official}$ is a current exchange rate, which is the ratio between foreign currency and dollars

                  \item If Big Mac Index $<$ 0

                    \begin{itemize}

                      \item PPP exchange rate < official exchange rate

                      \item Price level in foreign country is lower than we would expect if PPP held true

                      \item Real purchasing power in a foreign country is higher than in the US

                    \end{itemize}

                  \item If Big Mac Index $>$ 0

                    \begin{itemize}
                        
                      \item PPP exchange rate > official exchange rate

                      \item Price level in a foreign country is higher than if PPP held true

                      \item Real purchasing power in a foreign country is lower than in the US

                    \end{itemize}

                \end{enumerate}

            \end{enumerate}

          \item International Comparison Program (ICP) index is the main measure that economists actually use for international price comparisons, and is another measure of purchasing power index

            \begin{enumerate}

              \item Uses a broad market basket that tries to represent the full cost of living across countries

              \item Hard to create a market basket that tries to represent everywhere, since people in different countries consume different things depending on culture, climate, religion, etc.

              \item Still, imperfect PPP data is still better than no data at all

            \end{enumerate}

        \end{enumerate}

      \item Unemployment Rate

        \begin{enumerate}

          \item The last macroeconomic variable that measures the health of an economy is unemployment rate

          \item During a recession, which is a period of reduction in GDP, a lot of people lose their jobs due to significant reduction in production

          \item Even during expansion, which is a period of increase in GDP, there are people who temporarily have no job

        \end{enumerate}

      \item Effects of unemployment

        \begin{enumerate}

          \item The unemployed will lose their income for themselves and their families

          \item The unemployed waste their knowledge, skilles, and abilities

          \item The unemployed and their families may have psychological problems such as hopelessness, depression, and low self-esteem due to high uncertainty and the long period of joblessness

          \item The unemployed and their families may generate social problems such as high crime rates or suicide rates

          \item Total production of the economy will decline

        \end{enumerate}

      \item Labor Statistics

        \begin{enumerate}

          \item The Bureau of Labor Statistics (BLS) in the Department of Labor reports labor statistics monthly

          \item The BLS categorizes the working age population into three categories: labor force (the employed and unemployed), and people not in the labor force

          \item The BLS calculates three important labor statistics out of the three categories of the working age population: unemployment rate, labor participation rate, and the employment-population ratio

          \item There are tow major data sources of labor statistics: household survey and establishment survey

          \item The Current Population Survey (Household Survey) is a monthly survey of households conducted by the US Census Bureau on the charactersitics of people: the unemployed, employed, and people not in the labor force

          \item The Establishment Survey is a monthly survey of business establishments on the employed (number of persons employed) and a payroll

        \end{enumerate}

      \item Working-Age Population

        \begin{enumerate}

          \item The working-age population is a civilian non-institutional population aged 16 years and older

          \item Consists of th labor force (= the employed and unemployed) and people not in the labor force

          \item Exceptions: People living in correctional facilities, residential nursing facilities, mental health care facilities, and active military personnel

          \item The employed are those who have jobs (unless on a temporary vacation, personal leave, or illness during survey week) (Ex. Paid employees, self-employed, and unpaid family workers)

          \item The unemployed are those who have no jobs, are available for work, and are actively looking for work during 4 weeks prior to survey week (with the exception of those who are ill) (Ex. New entrants, job leavers, job losers, etc.)

          \item People not in the labor force are people who have no job, are not available for work, or are not actively looking for work during the 4 weeks prior to survey week (Ex. students, the elderly, or the disabled)
            
        \end{enumerate}

      \item Important Labor Statistics

        \begin{enumerate}

          \item Unemployment Rate (\%): $\frac{\text{Unemployed}}{\text{Labor Force}}\cdot100$

          \item Labor Force Participation Rate (\%): $\frac{\text{Labor Force}}{\text{Working-age Population}}\cdot100$

          \item Employment-Population Ratio (\%): $\frac{\text{Employed}}{\text{Working-age Population}}\cdot100$

        \end{enumerate}

      \item Unemployment Rates for Different Demographics

        \begin{enumerate}

          \item Until the start of the Great Recession (2007—2009), unemployment rates were the same for women or men. Afterwards, men were far more likely to be unemployed

          \item Younger people are more likely to be unemployed

          \item African-Americans have the highest rate, while Asians have the lowest. Latinos and Whites are in between.

          \item The more education a person has, the more likely he or she is to have a job

        \end{enumerate}

      \item The Labor Market

        \begin{enumerate}
            
          \item Just like all markets, a labor market has demand and supply

          \item The labor demand comes from firms, who need labor to produce products

          \item The labor demand curve shows the relationship between the quantity of labor demanded and the wage, which is the price of labor

          \item Firms want to hire more labor when wages are lower and less labor when wages are higher, ceteris paribus

          \item The labor demand curve is downward-sloping

          \item Labor supply comes from households who are able to work and choose to participate in the labor market

          \item The labor supply curve shows the relationship between the quantity of labor supplied and wage

          \item Households want to supply more labor at higher wage, and less labor at lower wage, ceteris paribus

          \item The labor supply curve is normally upward sloping

          \item Just like any market, the market equilibrium occurs at the intersection of supply and demand

          \item At equilibrium, the following happens:

            \begin{enumerate}

              \item The quantity demanded of labor equals the quantity supplied of labor

              \item Everyone who wants to work at the equilibrium wage can find a job and all firms who want to hire at the equilibrium wage can hire someone

            \end{enumerate}

          \item There are two types of market disequilibrium in the labor market:

            \begin{enumerate}

              \item At the wage above equilibrium, there is a surplus of labor, which means there is unemployment because some are unable to find a job

              \item With a surplus of labor, firms will lower the wage because there are more people who want to work, which makes the market return to equilibrium

              \item At the wage below equilibrium, there is a shortage of labor, because firms want to hire more workers than there are available

              \item With a shortage of labor, firms will raise the wage to attract more workers, until the market moves into equilibrium

              \item The labor market always returns to equilibrium through wage changes

            \end{enumerate}

        \end{enumerate}

      \item Categories of Unemployment

        \begin{enumerate}

          \item Unemployment rate rises during recession and falls during expansion, but there is always unemployment (even during expansion)

          \item There are two major categories of unemployment:

            \begin{enumerate}

              \item A natural rate of unemployment is a long run, normal level of unemployment that persists even during expansion, which consists of frictional, structural, and real wage (classical) unemployment and is a reason for a non-zero unemployment

                \begin{enumerate}

                  \item Historically, the US unemployment rate is always above 2\%, even during expansion

                  \item There are three types of natural unemployments, depending on the causes: frictional unemployment, structural unemployment, and real wage (classical) unemployment

                    \begin{itemize}

                      \item Frictional Unemployment — Occurs when it takes some time for workers to search for new openings, submit applications, and have interviews, in order to find better jobs or jobs in new locations

                      \item Structural Unemployment — Occurs when it takes some time for workers to receive training or education, search for new openings, submit applications and have interviews in order to find completely different types of jobs due to structural changes in industries

                      \item Real Wage (Classical) Unemployment — occurs when there is a surplus of labor at the wage above equilibrium wage due to minimum wage laws, bargaining by unions, and efficiency wage by firms, which prevent the wage from falling

                    \end{itemize}

                \end{enumerate}

              \item A cyclical unemployment is a short unemployment due to a business cycle

                \begin{enumerate}

                  \item A cyclical unemployment is caused by the short run fluctuations of the GDP, and is the unemployment that the government is most worried about

                  \item Recession causes labor demand to shift to the left because firms shrink their operations, resulting in a decrease of the equilibrium wage

                  \item This does not happen in the short run due to sticky wages, which occur due to long-term contracts on wages

                \end{enumerate}

            \end{enumerate}

        \end{enumerate}

      \item Issues of Unemployment

        \begin{enumerate}

          \item Factors that keep wages from falling 

            \begin{enumerate}

              \item Minimum Wage — The lowest wage possible, above the equilibrium wage, and set by the government to protect low-skilled or low educated workers. This generates a surplus of labor because the minimum wage is above the equilibrium wage. Workers who keep jobs at a higher wage will benefit, but workers who lose jobs and companies who need to pay more lose

              \item Labor Unions — A group of workers who join to bargain with their employers over salaries, health insurance, pensions, vacations, etc. This creates unemployment for the same reason as minimum wage. Workers who keep jobs will benefit, but workers who lose jobes will lose. Firms with unions cost more and have lower profits.

              \item Efficiency Wage — Occurs when some firms want to pay their workers higher wages than equilibrium because it makes workers less likely to quit, which saves the expense of hiring new workers, as well as makes workers more productive in order to keep their job. This generates efficiency, especially in sectors where worker motivation really matters. This may cause unemployment, though it is not definite.

            \end{enumerate}

        \end{enumerate}

      \item Other Factors that Influence Unemployment

        \begin{enumerate}

          \item Unemployment insurance is a payment by the government to workers who are unemployed or underemployed in order to provide insurance against a major loss of income

            \begin{enumerate}

              \item To receive unemployment insurance, there are certain conditions that need to be satisfied: active search for work, and reporting work-related activities

              \item The amount and duration of unemployment insurance varies across countries

              \item In the US, payment depends on how much an individual earned over the previous year. The standard duration of payment is 26 weeks

              \item This may lengthen the duration of unemployment and increase unemployment because it provides less incentive to urgently find a suitable job. It takes more time to find a perfect job, but reduces firctional unemployment in the long run

            \end{enumerate}

          \item Lower income tax gives people more incentive to find a job, whereas higher income tax gives people less incentive to find a job

          \item Policies to protect workers from firing or hiring would lead to greater unemployment

          \item The use of internships may reduce unemployment with fewer protections and salaries to employees

        \end{enumerate}

    \end{enumerate}

\end{document}

