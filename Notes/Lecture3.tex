%%%%%%%%%%%%%%%%%%%%%%%%%%%%%%%%%%%%%%%%%%%%%%%%%%%%%%%%%%%%%%%%%%%%%%%%%%%%%%%%%%%%%%%%%%%%%%%%%%%%%%%%%%%%%%%%%%%%%%%%%%%%%%%%%%%%%%%%%%%%%%%%%%%%%%%%%%%%%%%%%%%%%%%%%%%%%%%%%%%%%%%%%%%%
% Written By Michael Brodskiy
% Class: Principles of Macroeconomics
% Professor: H. Yoon
%%%%%%%%%%%%%%%%%%%%%%%%%%%%%%%%%%%%%%%%%%%%%%%%%%%%%%%%%%%%%%%%%%%%%%%%%%%%%%%%%%%%%%%%%%%%%%%%%%%%%%%%%%%%%%%%%%%%%%%%%%%%%%%%%%%%%%%%%%%%%%%%%%%%%%%%%%%%%%%%%%%%%%%%%%%%%%%%%%%%%%%%%%%%

\documentclass[12pt]{article} 
\usepackage{alphalph}
\usepackage[utf8]{inputenc}
\usepackage[russian,english]{babel}
\usepackage{titling}
\usepackage{amsmath}
\usepackage{graphicx}
\usepackage{enumitem}
\usepackage{amssymb}
\usepackage[super]{nth}
\usepackage{everysel}
\usepackage{ragged2e}
\usepackage{geometry}
\usepackage{fancyhdr}
\usepackage{cancel}
\usepackage{siunitx}
\usepackage{xcolor}
\usepackage{physics}
\usepackage{tikz}
\usepackage{mathdots}
\usepackage{yhmath}
\usepackage{color}
\usepackage{array}
\usepackage{multirow}
\usepackage{gensymb}
\usepackage{tabularx}
\usepackage{extarrows}
\usepackage{booktabs}
\usetikzlibrary{fadings}
\usetikzlibrary{patterns}
\usetikzlibrary{shadows.blur}
\usetikzlibrary{shapes}


\geometry{top=1.0in,bottom=1.0in,left=1.0in,right=1.0in}
\newcommand{\subtitle}[1]{%
  \posttitle{%
    \par\end{center}
    \begin{center}\large#1\end{center}
    \vskip0.5em}%

}
\usepackage{hyperref}
\hypersetup{
colorlinks=true,
linkcolor=blue,
filecolor=magenta,      
urlcolor=blue,
citecolor=blue,
}

\urlstyle{same}


\title{Lecture 3 Notes}
\date{July 7, 2021}
\author{Michael Brodskiy\\ \small Instructor: Prof. Yoon}

% Mathematical Operations:

% Sum: $$\sum_{n=a}^{b} f(x) $$
% Integral: $$\int_{lower}^{upper} f(x) dx$$
% Limit: $$\lim_{x\to\infty} f(x)$$

\begin{document}

    \maketitle

    \begin{enumerate}

      \item Aggregate Expenditure (AE)

        \begin{enumerate}

          \item John Maynard Keynes analyzed the short run relationship between the aggregate expenditure and GDP in his book, “\textit{The General Theory of Employment, Interest, and Money}” (1936)

          \item There is a fall in spending and production during a recession
            
          \item To explain the business cycle including recession, we must understand the components of aggregate expenditure

          \item The aggregate expenditure equation consists of four components:

            \begin{enumerate}

              \item $AE = C + I + G + NX$

              \item Aggregate expenditure is composed of expenditure by households ($C$), expenditure by firms ($I$), expenditure by government ($G$), and expenditure by foreigners minus domestic consumers ($NX$)

            \end{enumerate}

          \item The Components in Detail:

            \begin{enumerate}

              \item Consumption ($C$)

                \begin{enumerate}

                  \item An expenditure on goods and services by households and is the highest part of aggregate expenditure (70\%)

                  \item Consists of spending on nondurables, durables, and services

                  \item Factors that influence consumption:

                    \begin{itemize}

                      \item Disposable personal income (current income)

                        \begin{itemize}

                          \item DPI is the amount of income that is available for consumption after tax (adjustable income)

                          \item $DPI = PI - T$, where $PI$ is personal income, and $T$ is the income tax

                          \item Personal income = GDP + transfer payments + interest payments - retained earnings

                          \item Transfer payments are government spending on social welfare to households (ex. unemployment insurance, social security, disability insurance, etc.)

                          \item Interest payments are households' interest income from holding government bonds
                            
                          \item Retained earnings are the earnings from stocks that are reinvested to firms instead of paying back to stockholders as dividends

                          \item There is a positive relationship between DPI and consumption, ceteris paribus

                          \item The positive relationship between the two is shown as positive marginal propensity to consume

                          \item The marginal propensity to consume (MPC) is the increase in consumption as a result of increase in DPI by \$1, and ranges between 0 and 1

                        \end{itemize}

                      \item Wealth

                        \begin{itemize}

                          \item The value of assets minus the value of liabilities

                          \item Asset is anything of value owned by a person (ex. saving account, stocks, bonds, real estate, etc.)

                          \item Liability is anything of value owed by a person (ex. mortgage loans, car loans, credit card debts)

                          \item There is a positive relationship between wealth and consumption, ceteris paribus (with higher savings, households tend to consume more)

                        \end{itemize}
                        
                      \item Expected income (future income)

                        \begin{itemize}

                          \item If people expect their income to rise, they consume more from borrowing money or savings

                          \item If people expect their income to fall, they consume less in order to save money

                          \item There is a positive relationship between expected income and current consumption (ex. current consumption drops due to uncertainties from COVID-19)

                        \end{itemize}

                      \item Price level

                        \begin{itemize}

                          \item Inflation causes consumption to drop because people will have less purchasing power

                          \item Deflation leads to higher consumption because people will have higher purchasing powers

                          \item There is a negative relationship between price level and consumption

                        \end{itemize}

                      \item Interest rate

                        \begin{itemize}

                          \item Interest rate is a cost of borrowing and the price of money paid by borrowers

                          \item Interest rate is also a return on saving and the price of money received by savers

                          \item Increase in interest rate will raise the cost of borrowing and lower consumption, especially on durables

                          \item There is a negative relationship between interest rate and consumption

                        \end{itemize}

                    \end{itemize}

                \end{enumerate}

            \end{enumerate}

        \end{enumerate}

      \item Consumption Function

        \begin{enumerate}

          \item A function that shows the relationship between consumption and DPI ($Y$) assuming wealth, expected income, price level, and interest rate are constant (ceteris paribus)

          \item $C = a+bY$, where $a$ is the autonomous consumption (consumption when DPI = 0), and $b$ is the marginal propensity to consume (MPC), which is a change in consumption as a result of change in $DPI$ by \$1. $b=\frac{C_1-C_0}{Y_1-Y_0}=$ slope

          \item Assume that DPI = GDP ($Y$) or transfer payments + interest payments = retained earnings + income tax

        \end{enumerate}

      \item Investment

        \begin{enumerate}

          \item The expenditure on goods and services by firms

          \item It occupies similar proportion of aggregate expenditure as the government spending (20\%)

          \item It consists of non-residential fixed investment (structures, equipment, and intellectual properties), residential fixed investment (new houses bought by households), and change in inventories

          \item Factors that influence investment:

            \begin{enumerate}

              \item Expected profit (income)

                \begin{enumerate}

                  \item Firms will spend more money on non-residential investment if it is expected to earn higher profit

                  \item Households will spend more money on residential fixed investment if they are expected to have higher income

                  \item There is a positive relationship between expected profit or income and investment

                \end{enumerate}

              \item Interest rate

                \begin{enumerate}

                  \item Firms will borrow less money for non-residential fixed investment if the interest rate is high

                  \item Households will borrow less money for residential fixed investment if the interest rate is high

                  \item There is a negative relationship between interest rate and investment

                \end{enumerate}

              \item Property (corporate) tax

                \begin{enumerate}

                  \item Firms will spend less money on non-residential fixed investment if a corporate tax rises

                  \item Households will spend less money on residential fixed investment if the property tax or income tax rises

                  \item There is a negative relationship between tax and investment

                \end{enumerate}

              \item Investment credit

                \begin{enumerate}

                  \item Firms will spend more money on non-residential fixed investment if investment credit rises

                  \item Households spend more money on residential fixed investment if there is an investment credit for buying new houses

                  \item There is a positive relationship between investment credit and investment

                \end{enumerate}

              \item Cash flow

                \begin{enumerate}

                  \item The cash revenue minus cash cost

                  \item Profit is the largest portion

                  \item Depreciation is not a cash cost since firms are not paying for it by cash


                  \item Firms will spend more money on non-residential fixed investment if they have more cash flow

                  \item There is a positive relationship between cash flow and investment

                \end{enumerate}

              \item Price level

                \begin{enumerate}

                  \item Change in price level indirectly influences investment through change in interest rate

                  \item With inflation, people demand more money, which will result in increase in interest rate in the money market, and lower investment

                  \item With deflation, people demand less money, which will result in decrease in interest rate in the money market, and increase investment

                  \item There is a negative relationship between price level and investment

                \end{enumerate}

            \end{enumerate}

        \end{enumerate}

      \item Government purchases

        \begin{enumerate}

          \item An expenditure on goods and services by the government

          \item Occupies a similar proportion of aggregate expenditure as investment (20\%)

          \item Consists of spending by state, local, and federal governments

          \item The government decides how much it will spend according to citizens' needs

          \item In the short run, government purchases are not directly affected by income, wealth, interest rate, price level, and so on

          \item We assume that the government purchases are fixed over GDP or do not depend on GDP

          \item Transfer payments such as unemployment insurance and social security benefits are excluded in GDP because the government receives nothing in return

        \end{enumerate}

      \item Net export

        \begin{enumerate}

          \item It is exports minus imports

          \item Factors that influence net export:

            \begin{enumerate}

              \item Domestic income

                \begin{enumerate}

                  \item Increase in domestic income leads to an increase in import and decrease in net export

                  \item There is a negative relationship between domestic income and net export

                \end{enumerate}

              \item Foreign income

                \begin{enumerate}

                  \item Increase in foreign income leads to an increase in export and increase in net export

                  \item There is a positive relationship between foreign income and net export

                \end{enumerate}

              \item Exchange rate

                \begin{enumerate}

                  \item Exchange rate is the value of one currency expressed in terms of another currency, normally \$: foreign currency

                  \item Appreciation (increase in exchange rate)

                    \begin{itemize}

                      \item ex. from \$1:1000 Korean won to \$1:1200 Korean won

                      \item Domestic products become more expensive, and export will decrease

                      \item Foreign products become cheaper, and import will increase

                      \item As a result, net export will decrease

                    \end{itemize}

                  \item Depreciation (decrease in exchange rate)

                    \begin{itemize}

                      \item ex. from \$1:1000 Korean won to \$1:800 Korean won

                      \item Domestic products become cheaper, and export will increase

                      \item Foreign products become more expensive, and import will decrease

                      \item As a result, net export will increase

                    \end{itemize}

                  \item There is a negative relationship between exchange rate and net export

                \end{enumerate}

              \item Preferences for foreign goods

                \begin{enumerate}

                  \item When tastes for foreign goods increase or people find foreign goods more attractive, domestic consumers will buy more of foreign goods and import will increase and net export will decline

                  \item There is a negative relationship between tastes for foreign goods and net export

                \end{enumerate}

              \item Trade policies

                \begin{enumerate}

                  \item The effects of trade policies, such as free trade, tariffs, or import quota are reflected in exchange rates and other macroeconomic variables rather than influencing the net export directly

                  \item The effects of trade policies on net export can be analyzed on a case by case basis

                \end{enumerate}

              \item Price level

                \begin{enumerate}

                  \item Change in price level indirectly affects net export through change in exchange rate

                  \item With inflation, domestic products will become more expensive and exchange rate will rise (appreciate) because the value of domestic currency will increase, resulting in decrease in net export

                  \item With deflation, domestic products will become cheaper and exchange rate will decline (depreciate) because the value of domestic currency will decline, resulting in increase in net export

                \end{enumerate}

            \end{enumerate}

        \end{enumerate}

      \item Actual vs. Planned Aggregate Expenditure

        \begin{enumerate}

          \item Actual aggregate expenditure ($Y=$ GDP) is the amount of spending and production that the economy has already made, whereas planned aggregate expenditure (PAE) is the amount of spending and production that the economy is planning to make

          \item If PAE $>$ Y, there is a decrease in inventories because fewer products are produced than necessary

          \item If PAE $<$ Y, there is an increase in inventories because more products are produced than necessary

          \item If PAE $=$ Y, there is no change in inventories because products are produced exactly as needed

        \end{enumerate}

      \item Aggregate Expenditure Function

        \begin{enumerate}

          \item It is a function that shows the relationship between planned aggregate expenditure (PAE) and actual aggregate expenditure (GDP $=Y$)

          \item The aggregate expenditure is composed of four components: consumption, investment, government purchases, and net export

          \item $AE=C+I+G+NX$

          \item Consumption is composed of two parts: autonomous consumption and non-autonomous consumption — $C=a+bY$

          \item Manipulating the function gives us:

            \begin{enumerate}

              \item $AE = (a + I + G + NX) + bY\Rightarrow AE = d + bY$, where $d$ is the autonomous $AE$, which is an $AE$ that does not depend on income, and is composed of autonomous consumption, investment, government purchases, and net export (the Y-intercept of the curve), and $bY$ is the non-autonomous $AE$, which does depend on income, and is composed of non-autonomous consumption (slope of the curve)

            \end{enumerate}

        \end{enumerate}

      \item Macroeconomic (Keynesian) Equilibrium

        \begin{enumerate}

          \item Keynes explained the macroeconomic equilibrium using two lines: planned aggregate expenditure (PAE) and a 45 degree line that shows PAE = $Y$, which means whatever is produced in the economy is consumed (no change in inventories)

          \item Macroeconomic equilibrium occurs when there is no incentive to change or deviate in the economy

          \item It occurs when the PAE is equal to total production ($Y$) at the intersection between the $AE$ curve and a 45-degree line

          \item The GDP at this point is called macroeconomic equilibrium GDP

          \item If PAE $>$ Y, there is a decrease in inventories, and GDP will increase to equilibrium

          \item If PAE $<$ Y, there is an increase in inventories, and GDP will decrease to equilibrium

        \end{enumerate}

      \item Potential GDP and Business Cycle

        \begin{enumerate}

          \item Potential GDP or output ($Y_p$) is a GDP where there is no cyclical unemployment, a GDP where there is only the natural rate of unemployment, or a GDP at the full employment/capacity

          \item The most ideal GDP that an economy can reach

          \item The government uses this as a policy goal that the economy is supposed to reach

          \item About 4\% in the US, on average

          \item A business cycle (recession and expansion) occurs when the PAE curve shifts due to change in any non-income determinants of PAE (macroeconomic conditions)

            \begin{enumerate}

              \item Recession occurs when the equilibrium GDP ($Y_E$) $<$ potential GDP ($Y_p$)

              \item Occurs when any non-income determinants of PAE will shift the PAE curve right

              \item Occurs when wealth, expected profit, cash flow, or foreign income declines

              \item Occurs when interest rate, tax, exchange rate, tastes for foreign products, or price level rises

              \item The difference between equilibrium GDP and potential GDP is the recessionary output gap ($Y_p-Y_E$)

            \end{enumerate}

          \item Expansion occurs when the equilibrium GDP ($Y_E$) $>$ potential GDP ($Y_p$)

            \begin{enumerate}

              \item Occurs when any non-income determinants of PAE shift the PAE curve left

              \item Occurs when wealth, expected profit, cash flow, or foreign income rises

              \item Occurs when interest rate, tax, exchange rate, tastes for foreign products, or price level declines

              \item The difference between equilibrium GDP and potential GDP is the inflationary output gap ($Y_E-Y_p$)

            \end{enumerate}

        \end{enumerate}

      \item The Aggregate Demand and Aggregate Supply Model

        \begin{enumerate}

          \item Represents the relationship between GDP and price level for the economy, whereas demand and supply model represents the relationship between the price and quantity for a product

          \item GDP is the sum of the market values of all goods and services produced in the economy and represents the total production or total income for the economy

          \item The price level is the weighted average price of all the goods and services in the economy and is measured by GDP deflator, PPI, or CPI

        \end{enumerate}

      \item Aggregate Demand Curve

        \begin{enumerate}

          \item The aggregate demand is the total quantity of goods and services demanded in the economy at each given price level and is measured by the aggregate expenditure (AE) from the expenditure approach

          \item The aggregate demand curve is a curve that shows the relationship between price level and GDP

          \item The price level is measured by CPI, PPI, or GDP deflator, which is a weighted average price of all goods and services in the economy

          \item Changes in the price level influence the aggregate demand in three ways:

            \begin{enumerate}

              \item Wealth Effect (Real-Balance Effect)

                \begin{enumerate}

                  \item Change in the price level will influence consumption through change in the real value of people's money balance or purchasing power

                  \item Money balance includes checking and saving accounts, cash, and other dollar-denominated assets whose value does not change with change in the price level

                  \item Increase in the price level will reduce the purchasing power of money, which results in decrease in consumption, AE, and GDP

                  \item There is a negative relationship between price level and AE

                \end{enumerate}

              \item Interest Rate Effect

                \begin{enumerate}

                  \item Change in the price level will affect investment and consumption through change in the interest rate

                  \item Increase/decrease in the price level will increase/decrease the demand for money because it requires more/less money 

                  \item Increase/decrease in the money demand will rais/lower the interest rate (cost of borrowing) in the money market

                  \item With higher/lower cost of borrowing, there is a decrease/increase in consumption on durables, residential fixed investment, non-residential fixed investment, AE, and GDP

                  \item There is a negative relationship between price level and AE through change in interest rate

                \end{enumerate}

              \item Exchange Rate Effect (International Trade Effect)

                \begin{enumerate}

                  \item Changes in the price level will affect net export through change in exchange rates

                  \item Increase/decrease in the domestic price level will increase/decrease exchange rates through appreciation/depreciation of domestic currency

                  \item Appreciation/depreciation of domestic currency will increase/decrease import and decrease/increase export, resulting in decrease/increase in net export, AE, and Y

                  \item There is a negative relationship between price level and AE through change in exchange rate

                  \item Change in price level does not affect government purchases ($G$) because it is only affected by policy decisions, according to citizens' needs

                \end{enumerate}

              \item OVerall, there is a negative relationship between price level and AE through change in $C$, $I$, and $NX$ that comes from the wealth effect, interest rate effect, and exchange rate effect

              \item Thus, the aggregate demand is downward-sloping

            \end{enumerate}

          \item Non-price factors that shift the aggregate demand:

            \begin{enumerate}

              \item Increase in disposable personal income (DPI) will increase consumption and shift AD to the right

              \item Increase in wealth will increase consumption and shift AD to the right

              \item Increase in expected income will increase consumption and shift AD to the right

              \item Increase in interest rate will lower consumption by raising the cost of borrowing and shift AD to the left

            \end{enumerate}

          \item Non-price factors that affect investment:

            \begin{enumerate}

              \item Increase in expected income will increase the residential fixed investment and shift AD to the right

              \item Increase in expected profit will increase the non-residential fixed investment and shift AD to the right

              \item Increase in corporate tax will lower the non-residential fixed investment and shift AD to the left

              \item Increase in income or property tax will lower the residential fixed investment and shift AD to the left

              \item Increase in investment tax credit will increase non-residential fixed investment and shift AD to the right

              \item Increase in cash flow will increase the non-residential fixed investment and shift AD to the right

            \end{enumerate}

          \item Non-price factors that affect government purchases:

            \begin{enumerate}

              \item Expansionary fiscal policies will increase the government purchases and shift AD to the right

              \item Contractionary fiscal policy will lower the government purchaes and shift AD to the right

            \end{enumerate}

          \item Non-price factors that affect net export:

            \begin{enumerate}

              \item Higher domestic income will increase export and net export, and shift AD to the right

              \item Higher foreign income will increase import and decrease net export and shift AD to the left

              \item Higher exchange rate (appreciation of domestic currency) will reduce net export and shift AD to the left

              \item Higher tastes for foreign products will increase import and decrease net export and shift AD to the left

            \end{enumerate}

        \end{enumerate}

      \item Aggregate Supply Curve

        \begin{enumerate}

          \item The aggregate supply is the sum of the market values of all goods and services produced by all the firms in the economy

          \item The aggregate supply curve shows the relationship between price level and the value of total production by all firms in the economy

          \item Price level is again measured by CPI, GDP deflator, or PPI, and the value of total production is measured by GDP

          \item In macroeconomics, short run is different from long run in that input prices are sticky and not flexible in the short run

        \end{enumerate}

      \item Short Run Aggregate Supply (SRAS)

        \begin{enumerate}

          \item The short run aggregate supply curve shows the relationship between price level and GDP in the short run

          \item The short run aggregate supply curve is upward sloping because there is a positive relationship between price level and GDP

          \item Prices of products increase or decrease more quickly than the prices of inputs in the short run because the input prices are normally sticky and adjust slowly in response to change in output prices due to long term contracts of input prices, such as a 1-3 year wage contract

          \item Firms will produce more with higher profit because the prices of products increase more quickly than prices of inputs

          \item Firms will produce less with lower profit because the prices of products decrease more quickly than prices of inputs

          \item As a result, with higher price level, firms will increase production, but, with lower price level, will reduce production

        \end{enumerate}

      \item Long Run Aggregate Supply (LRAS)

        \begin{enumerate}

          \item Long-run in macroeconomics is not a set amount of time (it is the time it takes for prices of inputs to fully adjust to change in the prices of products)

          \item As a result of full adjustment of input prices to output prices, costs increase as much as revenues increases, resulting in the same profit as before price change

          \item Once input prices fully adjust to the new price level, the econo will go back to where it started

          \item In the long run, there is no efect of change in prices of produts on the aggregate supply

          \item Long run aggregate supply (LRAS) curve ir vertical and perfectly inelastic at the potential output ($Y_p$)

          \item The potential output is the output (GDP) at the full capacity, at the natural rate of unemployment, at full employment, or without cyclical unemployment

          \item The most ideal output and the greatest output produced by the economy by using all the resources at the full extent

          \item In conclusion, short run aggregate supply is upward sloping due to sticky input prices, but long run aggregate supply is vertical at the potential output due to full adjustment of input prices

        \end{enumerate}

      \item Non-price factors that shift SRAS and LRAS:

        \begin{enumerate}

          \item Increase in labor (human capital) will shift SRAS and LRAS to the right

          \item Increase in physical capital will shift SRAS and LRAS to the right

          \item Increase in natural resources will shift SRAS and LRAS to the right

          \item Positive technological change will shift SRAS and LRAS to the right, and vice versa

        \end{enumerate}

      \item Non-price factors that shift only SRAS:

        \begin{enumerate}

          \item Increase in the prices of inputs will shift SRAS to the left

          \item Increase in the expected prices of products will shift SRAS to the left because firms want to produce less now than in the future

          \item Increase in the expected prices of inputs will shift SRAS to the right because firms want to produce more now than in the future

          \item Changes in the prices of inputs and expected prices of inputs and outputs do not shift LRAS because, in the long run, input prices will fully adjust to change in prices of outputs, regardless of any changes in policies or expectations

          \item In the long run, the economy always goes back to the potential output, which is the most ideal economic output

        \end{enumerate}

      \item Macroeconomic Equilibrium

        \begin{enumerate}

          \item Short run macroeconomic equilibrium is a situation where there is no incentive to change or deviate in the economy in the short run

            \begin{enumerate}

              \item This occurs at the intersection between SRAS and AD

              \item Price level and GDP at the equilibrium are called short run equilibrium price level and GDP

            \end{enumerate}

          \item Long run macroeconomic equilibrium is a situation where there is no incentive to change or deviate in the economy in the long run

            \begin{enumerate}

              \item This occurs at the intersection among LRAS, SRAS, and AD (or at the potential GDP)

              \item Price level and GDP at this equilibrium is called long run equilibrium price level and GDP

            \end{enumerate}

          \item The economy is most ideal when short run equilibrium = long run equilibrium

        \end{enumerate}

      \item Business Cycle

        \begin{enumerate}

          \item Recession occurs when SRAS and AD intersect each other below the potential output or the short run equilibrium GDP is lower than $Y_p$

          \item The recessionary outgap is $Y_p-Y_{SR,E}$

          \item Expansion occurs when SRAS and AD intersect each other above the potential output or the short run equilibrium GDP is more than $Y_p$

          \item The inflationary outgap is $Y_{SR,E}-Y_p$

        \end{enumerate}

      \item Short Run Effects of Increase in AD:

        \begin{enumerate}

          \item The economy is initially in in the long run equilibrium with the equilibrium price level of $P_0$ and equilibrium GDP of $Y_p$

          \item Suppose that AD will shift to the right from $AD_0$ to $AD_1$ due to an increase in expected income

          \item The economy is in an expansion because the short run equilibrium $Y_{SR,E}>Y_p$

          \item Short run equilibrium price will rise to $P_1$

          \item There is an increase in both, price level and GDP in the short run

        \end{enumerate}

      \item Long Run Effects of Increase in AD:

        \begin{enumerate}

          \item Input prices will fully adjust to an increase in output prices in the long run

          \item This will shift SRAS to the left to SRAS$_1$ until the economy goes back to the potential output $Y_p$ at the new price level of $P_2$

          \item In the long run, there is no change in equilibrium GDP, but there is an increase in price level (inflation)

          \item In the long run, there is no change in nominal values, only in real values

        \end{enumerate}

      \item Effects of a Temporary Supply Shock

        \begin{enumerate}

          \item Temporary supply shock occurs when a non-price factor that shifts only SRAS changes

          \item Short Run Effects:

            \begin{enumerate}

              \item The economy is initially in the long run equilibrium with the equilibrium price level of $P_0$ and GDP of $Y_p$

              \item The increase in the price of oil (oil shock) will shift SRAS to the left from SRAS$_0$ to SRAS$_1$

              \item The economy is in recession because the short run equilibrium $Y_{SR,E}<Y_p$

              \item The short run equilibrium price level will rise to $P_1$

              \item This is called stagflation because there is stagnation along with inflation

            \end{enumerate}

          \item Long Run Effects:

            \begin{enumerate}

              \item Input prices will fully drop until the economy goes back to the potential output because input providers are willing to accept lower input prices due to low income and high unemployment during a recession, which will shift SRAS to SRAS$_0$

              \item In the long run, there is no change in equilibrium GDP and price level

            \end{enumerate}

        \end{enumerate}

      \item Effects of a Permanent Supply Shock

        \begin{enumerate}

          \item Permanent supply shocks occur when a non-price factor that shifts both LRAS and SRAS changes

            \begin{enumerate}

              \item Change in labor, human capital, physical capital, natural resources, and technology

            \end{enumerate}
            
          \item Short Run Effects:

            \begin{enumerate}

              \item The economy is initially in a long run equilibrium with equilibrium price level of $P_0$ and GDP of $Y_{p0}$

              \item The decrease will shift LRAS from LRAS$_0$ to LRAS$_1$ and shift SRAS from SRAS$_0$ to SRAS$_1$

              \item The economy is in an expansion because the short run equilibrium $Y_{SR,E}>Y_{p1}$

              \item The short run equilibrium price level rises to $P_1$

            \end{enumerate}

          \item Long Run Effects:

            \begin{enumerate}

              \item With an expansion, input prices will fully adjust to higher prices of produts in the long run until the economy goes back to the new potential GDP $Y_{p1}$

              \item This will shift SRAS to the left to SRAS$_2$

              \item The short run and long run equilibrium GDP will decline to the new potential GDP $Y_{p1}$

              \item The price level will increase to $P_2$

              \item In the long run, there is a decrease in equilibrium GDP and an increase in price level (inflation)

            \end{enumerate}

        \end{enumerate}

    \end{enumerate}

\end{document}

