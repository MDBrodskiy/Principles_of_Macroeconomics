%%%%%%%%%%%%%%%%%%%%%%%%%%%%%%%%%%%%%%%%%%%%%%%%%%%%%%%%%%%%%%%%%%%%%%%%%%%%%%%%%%%%%%%%%%%%%%%%%%%%%%%%%%%%%%%%%%%%%%%%%%%%%%%%%%%%%%%%%%%%%%%%%%%%%%%%%%%%%%%%%%%%%%%%%%%%%%%%%%%%%%%%%%%%
% Written By Michael Brodskiy
% Class: Principles of Macroeconomics
% Professor: H. Yoon
%%%%%%%%%%%%%%%%%%%%%%%%%%%%%%%%%%%%%%%%%%%%%%%%%%%%%%%%%%%%%%%%%%%%%%%%%%%%%%%%%%%%%%%%%%%%%%%%%%%%%%%%%%%%%%%%%%%%%%%%%%%%%%%%%%%%%%%%%%%%%%%%%%%%%%%%%%%%%%%%%%%%%%%%%%%%%%%%%%%%%%%%%%%%

\documentclass[12pt]{article} 
\usepackage{alphalph}
\usepackage[utf8]{inputenc}
\usepackage[russian,english]{babel}
\usepackage{titling}
\usepackage{amsmath}
\usepackage{graphicx}
\usepackage{enumitem}
\usepackage{amssymb}
\usepackage[super]{nth}
\usepackage{everysel}
\usepackage{ragged2e}
\usepackage{geometry}
\usepackage{fancyhdr}
\usepackage{cancel}
\usepackage{siunitx}
\usepackage{xcolor}
\usepackage{physics}
\usepackage{tikz}
\usepackage{mathdots}
\usepackage{yhmath}
\usepackage{color}
\usepackage{array}
\usepackage{multirow}
\usepackage{gensymb}
\usepackage{tabularx}
\usepackage{extarrows}
\usepackage{booktabs}
\usetikzlibrary{fadings}
\usetikzlibrary{patterns}
\usetikzlibrary{shadows.blur}
\usetikzlibrary{shapes}


\geometry{top=1.0in,bottom=1.0in,left=1.0in,right=1.0in}
\newcommand{\subtitle}[1]{%
  \posttitle{%
    \par\end{center}
    \begin{center}\large#1\end{center}
    \vskip0.5em}%

}
\usepackage{hyperref}
\hypersetup{
colorlinks=true,
linkcolor=blue,
filecolor=magenta,      
urlcolor=blue,
citecolor=blue,
}

\urlstyle{same}


\title{Lecture 3 Notes}
\date{July 7, 2021}
\author{Michael Brodskiy\\ \small Instructor: Prof. Yoon}

% Mathematical Operations:

% Sum: $$\sum_{n=a}^{b} f(x) $$
% Integral: $$\int_{lower}^{upper} f(x) dx$$
% Limit: $$\lim_{x\to\infty} f(x)$$

\begin{document}

    \maketitle

    \begin{enumerate}

      \item Aggregate Expenditure (AE)

        \begin{enumerate}

          \item John Maynard Keynes analyzed the short run relationship between the aggregate expenditure and GDP in his book, “\textit{The General Theory of Employment, Interest, and Money}” (1936)

          \item There is a fall in spending and production during a recession
            
          \item To explain the business cycle including recession, we must understand the components of aggregate expenditure

          \item The aggregate expenditure equation consists of four components:

            \begin{enumerate}

              \item $AE = C + I + G + NX$

              \item Aggregate expenditure is composed of expenditure by households ($C$), expenditure by firms ($I$), expenditure by government ($G$), and expenditure by foreigners minus domestic consumers ($NX$)

            \end{enumerate}

          \item The Components in Detail:

            \begin{enumerate}

              \item Consumption ($C$)

                \begin{enumerate}

                  \item An expenditure on goods and services by households and is the highest part of aggregate expenditure (70\%)

                  \item Consists of spending on nondurables, durables, and services

                  \item Factors that influence consumption:

                    \begin{itemize}

                      \item Disposable personal income (current income)

                        \begin{itemize}

                          \item DPI is the amount of income that is available for consumption after tax (adjustable income)

                          \item $DPI = PI - T$, where $PI$ is personal income, and $T$ is the income tax

                          \item Personal income = GDP + transfer payments + interest payments - retained earnings

                          \item Transfer payments are government spending on social welfare to households (ex. unemployment insurance, social security, disability insurance, etc.)

                          \item Interest payments are households' interest income from holding government bonds
                            
                          \item Retained earnings are the earnings from stocks that are reinvested to firms instead of paying back to stockholders as dividends

                          \item There is a positive relationship between DPI and consumption, ceteris paribus

                          \item The positive relationship between the two is shown as positive marginal propensity to consume

                          \item The marginal propensity to consume (MPC) is the increase in consumption as a result of increase in DPI by \$1, and ranges between 0 and 1

                        \end{itemize}

                      \item Wealth

                        \begin{itemize}

                          \item The value of assets minus the value of liabilities

                          \item Asset is anything of value owned by a person (ex. saving account, stocks, bonds, real estate, etc.)

                          \item Liability is anything of value owed by a person (ex. mortgage loans, car loans, credit card debts)

                          \item There is a positive relationship between wealth and consumption, ceteris paribus (with higher savings, households tend to consume more)

                        \end{itemize}
                        
                      \item Expected income (future income)

                        \begin{itemize}

                          \item If people expect their income to rise, they consume more from borrowing money or savings

                          \item If people expect their income to fall, they consume less in order to save money

                          \item There is a positive relationship between expected income and current consumption (ex. current consumption drops due to uncertainties from COVID-19)

                        \end{itemize}

                      \item Price level

                        \begin{itemize}

                          \item Inflation causes consumption to drop because people will have less purchasing power

                          \item Deflation leads to higher consumption because people will have higher purchasing powers

                          \item There is a negative relationship between price level and consumption

                        \end{itemize}

                      \item Interest rate

                        \begin{itemize}

                          \item Interest rate is a cost of borrowing and the price of money paid by borrowers

                          \item Interest rate is also a return on saving and the price of money received by savers

                          \item Increase in interest rate will raise the cost of borrowing and lower consumption, especially on durables

                          \item There is a negative relationship between interest rate and consumption

                        \end{itemize}

                    \end{itemize}

                \end{enumerate}

            \end{enumerate}

        \end{enumerate}

      \item Consumption Function

        \begin{enumerate}

          \item A function that shows the relationship between consumption and DPI ($Y$) assuming wealth, expected income, price level, and interest rate are constant (ceteris paribus)

          \item $C = a+bY$, where $a$ is the autonomous consumption (consumption when DPI = 0), and $b$ is the marginal propensity to consume (MPC), which is a change in consumption as a result of change in $DPI$ by \$1. $b=\frac{C_1-C_0}{Y_1-Y_0}=$ slope

          \item Assume that DPI = GDP ($Y$) or transfer payments + interest payments = retained earnings + income tax

        \end{enumerate}

      \item Investment

        \begin{enumerate}

          \item The expenditure on goods and services by firms

          \item It occupies similar proportion of aggregate expenditure as the government spending (20\%)

          \item It consists of non-residential fixed investment (structures, equipment, and intellectual properties), residential fixed investment (new houses bought by households), and change in inventories

          \item Factors that influence investment:

            \begin{enumerate}

              \item Expected profit (income)

                \begin{enumerate}

                  \item Firms will spend more money on non-residential investment if it is expected to earn higher profit

                  \item Households will spend more money on residential fixed investment if they are expected to have higher income

                  \item There is a positive relationship between expected profit or income and investment

                \end{enumerate}

              \item Interest rate

                \begin{enumerate}

                  \item Firms will borrow less money for non-residential fixed investment if the interest rate is high

                  \item Households will borrow less money for residential fixed investment if the interest rate is high

                  \item There is a negative relationship between interest rate and investment

                \end{enumerate}

              \item Property (corporate) tax

                \begin{enumerate}

                  \item Firms will spend less money on non-residential fixed investment if a corporate tax rises

                  \item Households will spend less money on residential fixed investment if the property tax or income tax rises

                  \item There is a negative relationship between tax and investment

                \end{enumerate}

              \item Investment credit

                \begin{enumerate}

                  \item Firms will spend more money on non-residential fixed investment if investment credit rises

                  \item Households spend more money on residential fixed investment if there is an investment credit for buying new houses

                  \item There is a positive relationship between investment credit and investment

                \end{enumerate}

              \item Cash flow

                \begin{enumerate}

                  \item The cash revenue minus cash cost

                  \item Profit is the largest portion

                  \item Depreciation is not a cash cost since firms are not paying for it by cash


                  \item Firms will spend more money on non-residential fixed investment if they have more cash flow

                  \item There is a positive relationship between cash flow and investment

                \end{enumerate}

              \item Price level

                \begin{enumerate}

                  \item Change in price level indirectly influences investment through change in interest rate

                  \item With inflation, people demand more money, which will result in increase in interest rate in the money market, and lower investment

                  \item With deflation, people demand less money, which will result in decrease in interest rate in the money market, and increase investment

                  \item There is a negative relationship between price level and investment

                \end{enumerate}

            \end{enumerate}

        \end{enumerate}

      \item Government purchases

        \begin{enumerate}

          \item An expenditure on goods and services by the government

          \item Occupies a similar proportion of aggregate expenditure as investment (20\%)

          \item Consists of spending by state, local, and federal governments

          \item The government decides how much it will spend according to citizens' needs

          \item In the short run, government purchases are not directly affected by income, wealth, interest rate, price level, and so on

          \item We assume that the government purchases are fixed over GDP or do not depend on GDP

          \item Transfer payments such as unemployment insurance and social security benefits are excluded in GDP because the government receives nothing in return

        \end{enumerate}

      \item Net export

        \begin{enumerate}

          \item It is exports minus imports

          \item Factors that influence net export:

            \begin{enumerate}

              \item Domestic income

                \begin{enumerate}

                  \item Increase in domestic income leads to an increase in import and decrease in net export

                  \item There is a negative relationship between domestic income and net export

                \end{enumerate}

              \item Foreign income

                \begin{enumerate}

                  \item Increase in foreign income leads to an increase in export and increase in net export

                  \item There is a positive relationship between foreign income and net export

                \end{enumerate}

              \item Exchange rate

                \begin{enumerate}

                  \item Exchange rate is the value of one currency expressed in terms of another currency, normally \$: foreign currency

                  \item Appreciation (increase in exchange rate)

                    \begin{itemize}

                      \item ex. from \$1:1000 Korean won to \$1:1200 Korean won

                      \item Domestic products become more expensive, and export will decrease

                      \item Foreign products become cheaper, and import will increase

                      \item As a result, net export will decrease

                    \end{itemize}

                  \item Depreciation (decrease in exchange rate)

                    \begin{itemize}

                      \item ex. from \$1:1000 Korean won to \$1:800 Korean won

                      \item Domestic products become cheaper, and export will increase

                      \item Foreign products become more expensive, and import will decrease

                      \item As a result, net export will increase

                    \end{itemize}

                  \item There is a negative relationship between exchange rate and net export

                \end{enumerate}

              \item Preferences for foreign goods

                \begin{enumerate}

                  \item When tastes for foreign goods increase or people find foreign goods more attractive, domestic consumers will buy more of foreign goods and import will increase and net export will decline

                  \item There is a negative relationship between tastes for foreign goods and net export

                \end{enumerate}

              \item Trade policies

                \begin{enumerate}

                  \item The effects of trade policies, such as free trade, tariffs, or import quota are reflected in exchange rates and other macroeconomic variables rather than influencing the net export directly

                  \item The effects of trade policies on net export can be analyzed on a case by case basis

                \end{enumerate}

              \item Price level

                \begin{enumerate}

                  \item Change in price level indirectly affects net export through change in exchange rate

                  \item With inflation, domestic products will become more expensive and exchange rate will rise (appreciate) because the value of domestic currency will increase, resulting in decrease in net export

                  \item With deflation, domestic products will become cheaper and exchange rate will decline (depreciate) because the value of domestic currency will decline, resulting in increase in net export

                \end{enumerate}

            \end{enumerate}

        \end{enumerate}

      \item Actual vs. Planned Aggregate Expenditure

        \begin{enumerate}

          \item Actual aggregate expenditure ($Y=$ GDP) is the amount of spending and production that the economy has already made, whereas planned aggregate expenditure (PAE) is the amount of spending and production that the economy is planning to make

          \item If PAE $>$ Y, there is a decrease in inventories because fewer products are produced than necessary

          \item If PAE $<$ Y, there is an increase in inventories because more products are produced than necessary

          \item If PAE $=$ Y, there is no change in inventories because products are produced exactly as needed

        \end{enumerate}

      \item Aggregate Expenditure Function

        \begin{enumerate}

          \item It is a function that shows the relationship between planned aggregate expenditure (PAE) and actual aggregate expenditure (GDP $=Y$)

          \item The aggregate expenditure is composed of four components: consumption, investment, government purchases, and net export

          \item $AE=C+I+G+NX$

          \item Consumption is composed of two parts: autonomous consumption and non-autonomous consumption — $C=a+bY$

          \item Manipulating the function gives us:

            \begin{enumerate}

              \item $AE = (a + I + G + NX) + bY\Rightarrow AE = d + bY$, where $d$ is the autonomous $AE$, which is an $AE$ that does not depend on income, and is composed of autonomous consumption, investment, government purchases, and net export (the Y-intercept of the curve), and $bY$ is the non-autonomous $AE$, which does depend on income, and is composed of non-autonomous consumption (slope of the curve)

            \end{enumerate}

        \end{enumerate}

      \item Macroeconomic (Keynesian) Equilibrium

        \begin{enumerate}

          \item Keynes explained the macroeconomic equilibrium using two lines: planned aggregate expenditure (PAE) and a 45 degree line that shows PAE = $Y$, which means whatever is produced in the economy is consumed (no change in inventories)

          \item Macroeconomic equilibrium occurs when there is no incentive to change or deviate in the economy

          \item It occurs when the PAE is equal to total production ($Y$) at the intersection between the $AE$ curve and a 45-degree line

          \item The GDP at this point is called macroeconomic equilibrium GDP

          \item If PAE $>$ Y, there is a decrease in inventories, and GDP will increase to equilibrium

          \item If PAE $<$ Y, there is an increase in inventories, and GDP will decrease to equilibrium

        \end{enumerate}

      \item Potential GDP and Business Cycle

        \begin{enumerate}

          \item Potential GDP or output ($Y_p$) is a GDP where there is no cyclical unemployment, a GDP where there is only the natural rate of unemployment, or a GDP at the full employment/capacity

          \item The most ideal GDP that an economy can reach

          \item The government uses this as a policy goal that the economy is supposed to reach

          \item About 4\% in the US, on average

          \item A business cycle (recession and expansion) occurs when the PAE curve shifts due to change in any non-income determinants of PAE (macroeconomic conditions)

            \begin{enumerate}

              \item Recession occurs when the equilibrium GDP ($Y_E$) $<$ potential GDP ($Y_p$)

              \item Occurs when any non-income determinants of PAE will shift the PAE curve right

              \item Occurs when wealth, expected profit, cash flow, or foreign income declines

              \item Occurs when interest rate, tax, exchange rate, tastes for foreign products, or price level rises

              \item The difference between equilibrium GDP and potential GDP is the recessionary output gap ($Y_p-Y_E$)

            \end{enumerate}

          \item Expansion occurs when the equilibrium GDP ($Y_E$) $>$ potential GDP ($Y_p$)

            \begin{enumerate}

              \item Occurs when any non-income determinants of PAE shift the PAE curve left

              \item Occurs when wealth, expected profit, cash flow, or foreign income rises

              \item Occurs when interest rate, tax, exchange rate, tastes for foreign products, or price level declines

              \item The difference between equilibrium GDP and potential GDP is the inflationary output gap ($Y_E-Y_p$)

            \end{enumerate}

        \end{enumerate}

      \item The Aggregate Demand and Aggregate Supply Model

        \begin{enumerate}

          \item Represents the relationship between GDP and price level for the economy, whereas demand and supply model represents the relationship between the price and quantity for a product

          \item GDP is the sum of the market values of all goods and services produced in the economy and represents the total production or total income for the economy

          \item The price level is the weighted average price of all the goods and services in the economy and is measured by GDP deflator, PPI, or CPI

        \end{enumerate}

      \item Aggregate Demand Curve

        \begin{enumerate}

          \item The aggregate demand is the total quantity of goods and services demanded in the economy at each given price level and is measured by the aggregate expenditure (AE) from the expenditure approach

          \item The aggregate demand curve is a curve that shows the relationship between price level and GDP

          \item The price level is measured by CPI, PPI, or GDP deflator, which is a weighted average price of all goods and services in the economy

          \item Changes in the price level influence the aggregate demand in three ways:

            \begin{enumerate}

              \item Wealth Effect (Real-Balance Effect)

              \item Interest Rate Effect

              \item Exchange Rate Effect (International Trade Effect)

            \end{enumerate}

        \end{enumerate}

    \end{enumerate}

\end{document}

