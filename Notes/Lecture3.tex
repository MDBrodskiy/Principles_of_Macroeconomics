%%%%%%%%%%%%%%%%%%%%%%%%%%%%%%%%%%%%%%%%%%%%%%%%%%%%%%%%%%%%%%%%%%%%%%%%%%%%%%%%%%%%%%%%%%%%%%%%%%%%%%%%%%%%%%%%%%%%%%%%%%%%%%%%%%%%%%%%%%%%%%%%%%%%%%%%%%%%%%%%%%%%%%%%%%%%%%%%%%%%%%%%%%%%
% Written By Michael Brodskiy
% Class: Principles of Macroeconomics
% Professor: H. Yoon
%%%%%%%%%%%%%%%%%%%%%%%%%%%%%%%%%%%%%%%%%%%%%%%%%%%%%%%%%%%%%%%%%%%%%%%%%%%%%%%%%%%%%%%%%%%%%%%%%%%%%%%%%%%%%%%%%%%%%%%%%%%%%%%%%%%%%%%%%%%%%%%%%%%%%%%%%%%%%%%%%%%%%%%%%%%%%%%%%%%%%%%%%%%%

\documentclass[12pt]{article} 
\usepackage{alphalph}
\usepackage[utf8]{inputenc}
\usepackage[russian,english]{babel}
\usepackage{titling}
\usepackage{amsmath}
\usepackage{graphicx}
\usepackage{enumitem}
\usepackage{amssymb}
\usepackage[super]{nth}
\usepackage{everysel}
\usepackage{ragged2e}
\usepackage{geometry}
\usepackage{fancyhdr}
\usepackage{cancel}
\usepackage{siunitx}
\usepackage{xcolor}
\usepackage{physics}
\usepackage{tikz}
\usepackage{mathdots}
\usepackage{yhmath}
\usepackage{color}
\usepackage{array}
\usepackage{multirow}
\usepackage{gensymb}
\usepackage{tabularx}
\usepackage{extarrows}
\usepackage{booktabs}
\usetikzlibrary{fadings}
\usetikzlibrary{patterns}
\usetikzlibrary{shadows.blur}
\usetikzlibrary{shapes}


\geometry{top=1.0in,bottom=1.0in,left=1.0in,right=1.0in}
\newcommand{\subtitle}[1]{%
  \posttitle{%
    \par\end{center}
    \begin{center}\large#1\end{center}
    \vskip0.5em}%

}
\usepackage{hyperref}
\hypersetup{
colorlinks=true,
linkcolor=blue,
filecolor=magenta,      
urlcolor=blue,
citecolor=blue,
}

\urlstyle{same}


\title{Lecture 3 Notes}
\date{July 7, 2021}
\author{Michael Brodskiy\\ \small Instructor: Prof. Yoon}

% Mathematical Operations:

% Sum: $$\sum_{n=a}^{b} f(x) $$
% Integral: $$\int_{lower}^{upper} f(x) dx$$
% Limit: $$\lim_{x\to\infty} f(x)$$

\begin{document}

    \maketitle

    \begin{enumerate}

      \item Aggregate Expenditure (AE)

        \begin{enumerate}

          \item John Maynard Keynes analyzed the short run relationship between the aggregate expenditure and GDP in his book, “\textit{The General Theory of Employment, Interest, and Money}” (1936)

          \item There is a fall in spending and production during a recession
            
          \item To explain the business cycle including recession, we must understand the components of aggregate expenditure

          \item The aggregate expenditure equation consists of four components:

            \begin{enumerate}

              \item $AE = C + I + G + NX$

              \item Aggregate expenditure is composed of expenditure by households ($C$), expenditure by firms ($I$), expenditure by government ($G$), and expenditure by foreigners minus domestic consumers ($NX$)

            \end{enumerate}

          \item The Components in Detail:

            \begin{enumerate}

              \item Consumption ($C$)

                \begin{enumerate}

                  \item An expenditure on goods and services by households and is the highest part of aggregate expenditure (70\%)

                  \item Consists of spending on nondurables, durables, and services

                  \item Factors that influence consumption:

                    \begin{itemize}

                      \item Disposable personal income (current income)

                        \begin{itemize}

                          \item DPI is the amount of income that is available for consumption after tax (adjustable income)

                          \item $DPI = PI - T$, where $PI$ is personal income, and $T$ is the income tax

                          \item Personal income = GDP + transfer payments + interest payments - retained earnings

                          \item Transfer payments are government spending on social welfare to households (ex. unemployment insurance, social security, disability insurance, etc.)

                          \item Interest payments are households' interest income from holding government bonds
                            
                          \item Retained earnings are the earnings from stocks that are reinvested to firms instead of paying back to stockholders as dividends

                          \item There is a positive relationship between DPI and consumption, ceteris paribus

                          \item The positive relationship between the two is shown as positive marginal propensity to consume

                          \item The marginal propensity to consume (MPC) is the increase in consumption as a result of increase in DPI by \$1, and ranges between 0 and 1

                        \end{itemize}

                      \item Wealth

                        \begin{itemize}

                          \item The value of assets minus the value of liabilities

                          \item Asset is anything of value owned by a person (ex. saving account, stocks, bonds, real estate, etc.)

                          \item Liability is anything of value owed by a person (ex. mortgage loans, car loans, credit card debts)

                          \item There is a positive relationship between wealth and consumption, ceteris paribus (with higher savings, households tend to consume more)

                        \end{itemize}
                        
                      \item Expected income (future income)

                      \item Price level

                      \item Interest rate

                    \end{itemize}

                \end{enumerate}

            \end{enumerate}

        \end{enumerate}

    \end{enumerate}

\end{document}

